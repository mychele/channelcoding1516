\documentclass[10pt]{article}

%% Various useful packages and commands from different sources

\usepackage[applemac]{inputenc}
\usepackage[english]{babel}
\usepackage[T1]{fontenc}
\usepackage{cite, url,color} % Citation numbers being automatically sorted and properly "compressed/ranged".
%\usepackage{pgfplots}
\usepackage{graphics,amsfonts}
\usepackage[pdftex]{graphicx}
\usepackage[cmex10]{amsmath}
% Also, note that the amsmath package sets \interdisplaylinepenalty to 10000
% thus preventing page breaks from occurring within multiline equations. Use:
 \interdisplaylinepenalty=2500
% after loading amsmath to restore such page breaks as IEEEtran.cls normally does.

% Compact lists
\usepackage{enumitem}
\usepackage{booktabs}
\usepackage{fancyvrb}

\usepackage{listings} % for Matlab code
\definecolor{commenti}{rgb}{0.13,0.55,0.13}
\definecolor{stringhe}{rgb}{0.63,0.125,0.94}
\lstloadlanguages{C, MATLAB}
\lstset{% general command to set parameter(s)
% framexleftmargin=0mm,
% frame=single,
keywordstyle = \color{blue},% blue keywords
% identifierstyle =, % nothing happens
commentstyle = \color{commenti}, % comments
stringstyle = \ttfamily \color{stringhe}, % typewriter type for strings
% showstringspaces = false, % no special string spaces
emph = {for, if, then, else, end},
% emphstyle = \color{blue},
% firstnumber = 1,
numbers =right, %  show number_line
numberstyle = \tiny, % style of number_line
stepnumber = 5, % one number_line after stepnumber
numbersep = 5pt,
language = {C},
% extendedchars = true,
% breaklines = true,
% breakautoindent = true,
% breakindent = 30pt,
keepspaces=true,                 % keeps spaces in text, useful for keeping indentation of code (possibly needs columns=flexible)
basicstyle=\footnotesize\ttfamily,
tabsize = 4
}

\lstset{% general command to set parameter(s)
% framexleftmargin=0mm,
% frame=single,
keywordstyle = \color{blue},% blue keywords
% identifierstyle =, % nothing happens
commentstyle = \color{commenti}, % comments
stringstyle = \ttfamily \color{stringhe}, % typewriter type for strings
showstringspaces = false, % no special string spaces
emph = {for, if, then, else, end},
% emphstyle = \color{blue},
% firstnumber = 1,
numbers =right, %  show number_line
numberstyle = \tiny, % style of number_line
stepnumber = 5, % one number_line after stepnumber
numbersep = 5pt,
language = {MATLAB},
extendedchars = true,
% breaklines = true,
% breakautoindent = true,
% breakindent = 30pt,
keepspaces=true,                 % keeps spaces in text, useful for keeping indentation of code (possibly needs columns=flexible)
basicstyle=\footnotesize\ttfamily,
tabsize = 4
}

\usepackage{array}
% http://www.ctan.org/tex-archive/macros/latex/required/tools/
\usepackage{mdwmath}
\usepackage{mdwtab}
%mdwtab.sty	-- A complete ground-up rewrite of LaTeX's `tabular' and  `array' environments.  Has lots of advantages over
%		   the standard version, and over the version in `array.sty'.
% *** SUBFIGURE PACKAGES ***
\usepackage[tight,footnotesize]{subfigure}
\usepackage[top=2cm, bottom=2cm, right=1.6cm,left=1.6cm]{geometry}
\usepackage{indentfirst}


\setlength\parindent{0pt}
\linespread{1}

\usepackage{mathtools}
\DeclarePairedDelimiter{\ceil}{\lceil}{\rceil}
\DeclarePairedDelimiter{\floor}{\lfloor}{\rfloor}
\DeclareMathOperator*{\argmax}{arg\,max}


% equations are numbered section by section
\numberwithin{equation}{section}


\begin{document}
\title{Channel Coding 15/16 - Homework 1}
\author{Michele Polese}

\maketitle

A live version of the following code and all its little variations can be found on my Github repository.

\section*{C code}

This is the code that implements a Viterbi Decoder for a (5,1,7) binary convolutional code, both with soft decoding (SD) and hard decoding (HD) options. The code can be easily adapted to decode a (5, 7, 7) binary convolutional code by swapping the \texttt{y\_lut} output lookup table.
\subsection*{\texttt{viterbi517.h}}
\lstinputlisting[language=C]{../viterbi517.h}
\subsection*{\texttt{viterbi517.c}}
\lstinputlisting[language=C]{../viterbi517.c}

The following code implements a windowed version of Viterbi decoding for a (5,1,7) binary convolutional code. The window size is left as an option. It relies on the viterbi517.h methods to compute the cost of the transitions.

\subsection*{\texttt{viterbi517\_windowed.h}}
\lstinputlisting[language=C]{../viterbi517_windowed.h}
\clearpage
\subsection*{\texttt{viterbi517\_windowed.c}}
\lstinputlisting[language=C]{../viterbi517_windowed.c}

This is the implementation of an encoder for (5,1,7) binary convolutional code.

\subsection*{\texttt{encoder517.h}}
\lstinputlisting[language=C]{../encoder517.h}
\subsection*{\texttt{encoder517.c}}
\lstinputlisting[language=C]{../encoder517.c}

This is an example of a MEX function that calls the Viterbi decoder for (5,1,7) from MATLAB. Different MEX function were written to call the windowed version and the encoder, but since they are very similar to this are not reported here.

\subsection*{\texttt{viterbi\_mex.c}}
\lstinputlisting[language=C]{../viterbi_mex.c}

\section*{MATLAB code}

\subsection*{\texttt{cc\_hw1.m}}
This is the main script for the first exercise. It executes all the simulations needed automatically, there is an option to store results into a .mat file, and outputs different plots. The parameters of the simulations are set in the first lines of code, but they can be tweaked for each different simulation.
\lstinputlisting{../cc_hw1.m}

\subsection*{\texttt{cc\_hw1\_e5.m}}
This is the script for exercise 5. It uses the function \texttt{sum\_rows.m} to perform sum between rows in $\mathbb{F}_2$.
\lstinputlisting{../cc_hw1_e5.m}
\subsection*{\texttt{sum\_rows.m}}
\lstinputlisting{../sum_rows.m}

\subsection*{\texttt{dmin\_finder.m}} 
In this script I implemented the algorithm described in exercise 7 to find the $d_{min}$ of a convolutional code.
\lstinputlisting{../dmin_finder.m}
\end{document}
