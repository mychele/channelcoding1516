\documentclass[10pt]{article}

%% Various useful packages and commands from different sources

\usepackage[applemac]{inputenc}
\usepackage[english]{babel}
\usepackage[T1]{fontenc}
\usepackage{cite, url,color} % Citation numbers being automatically sorted and properly "compressed/ranged".
%\usepackage{pgfplots}
\usepackage{graphics,amsfonts}
\usepackage[pdftex]{graphicx}
\usepackage[cmex10]{amsmath}
\usepackage{amssymb}
\usepackage{bm}
% Also, note that the amsmath package sets \interdisplaylinepenalty to 10000
% thus preventing page breaks from occurring within multiline equations. Use:
 \interdisplaylinepenalty=2500
% after loading amsmath to restore such page breaks as IEEEtran.cls normally does.

% Compact lists
\usepackage{enumitem}
\usepackage{booktabs}
\usepackage{fancyvrb}

% Tikz
\usepackage{tikz}
\usetikzlibrary{automata,positioning,chains,shapes,arrows}
\usepackage{pgfplots}
\usetikzlibrary{plotmarks}
\newlength\fheight
\newlength\fwidth
\pgfplotsset{compat=newest}
\pgfplotsset{plot coordinates/math parser=false}

\usepackage{listings} % for Matlab code
\definecolor{commenti}{rgb}{0.13,0.55,0.13}
\definecolor{stringhe}{rgb}{0.63,0.125,0.94}
\lstloadlanguages{Matlab}
\lstset{% general command to set parameter(s)
framexleftmargin=0mm,
frame=single,
keywordstyle = \color{blue},% blue keywords
identifierstyle =, % nothing happens
commentstyle = \color{commenti}, % comments
stringstyle = \ttfamily \color{stringhe}, % typewriter type for strings
showstringspaces = false, % no special string spaces
emph = {for, if, then, else, end},
emphstyle = \color{blue},
firstnumber = 1,
numbers =right, %  show number_line
numberstyle = \tiny, % style of number_line
stepnumber = 5, % one number_line after stepnumber
numbersep = 5pt,
language = {Matlab},
extendedchars = true,
breaklines = true,
breakautoindent = true,
breakindent = 30pt,
basicstyle=\footnotesize\ttfamily
}

\usepackage{array}
% http://www.ctan.org/tex-archive/macros/latex/required/tools/
\usepackage{mdwmath}
\usepackage{mdwtab}
%mdwtab.sty	-- A complete ground-up rewrite of LaTeX's `tabular' and  `array' environments.  Has lots of advantages over
%		   the standard version, and over the version in `array.sty'.
% *** SUBFIGURE PACKAGES ***
\usepackage[tight,footnotesize]{subfigure}
\usepackage[top=2.2cm, bottom=2.2cm, right=1.7cm,left=1.7cm]{geometry}
\usepackage{indentfirst}


%\setlength\parindent{0pt}
\linespread{1}

\usepackage{mathtools}
\DeclarePairedDelimiter{\ceil}{\lceil}{\rceil}
\DeclarePairedDelimiter{\floor}{\lfloor}{\rfloor}
\DeclareMathOperator*{\argmax}{arg\,max}
\newcommand{\M} {\mathtt{M}}
\newcommand{\dB} {\mathrm{dB}}
\newcommand{\tr} {\mathrm{tr}}
\newcommand{\lmod}[1] {_{\,\mathrm{mod}\,#1}}
\newcommand{\outf}[1] {\mathcal{O}(#1)}
\newcommand{\SU}[1] {\mathcal{S}(#1)}
\newcommand{\s} {\mathbf{s}}
\newcommand{\y} {\mathbf{y}}


\graphicspath{ {figures/} }

% equations are numbered section by section
%\numberwithin{equation}{section}


\begin{document}
\title{Channel Coding 15/16 - Homework 1}
\author{Michele Polese}

\maketitle

% For tikz
% Definition of blocks:
\tikzstyle{block} = [draw, rectangle, 
    minimum height=3em, minimum width=3em]
\tikzstyle{circlenode} = [draw, circle, minimum height=3em, minimum width=3em]
\tikzstyle{sum} = [draw, circle, node distance=1cm]
\tikzstyle{input} = [coordinate]
\tikzstyle{output} = [coordinate]
\tikzstyle{pinstyle} = [pin edge={to-,thin,black}]

%%%%%%%%%%%%%%%%%%%%%%%%%%%%%%%%%%%%%%%%%
%%%%%%%%%%%%%% PROBLEM 1 %%%%%%%%%%%%%%%%
%%%%%%%%%%%%%%%%%%%%%%%%%%%%%%%%%%%%%%%%%

\section*{Exercise 1}
In the first exercise I will provide an analysis of the performances of the binary convolutional code defined by the triplet 
\begin{equation}
	\mathbf{g} = \mathbf{g}_3 = \begin{bmatrix}
			1 + D^2 \\
			D^2	\\
			1 + D + D^2 \\
			\end{bmatrix}
\end{equation}
This is a polynomial code, whose realization diagram is in Fig.~\ref{fig:diag}. Its memory is $\nu = 2$ and since it is a binary code with $q = 2$ the state space has size $q^\nu = 4$. Note that every sum from now on will be a sum in $\mathbb{F}_2$ (i.e. modulo 2 sum) unless specified.

\begin{figure}[h]
\centering

% picture of realization diagram
\begin{tikzpicture}[auto, thick, node distance=3cm, >=latex']
	% Drawing the blocks of first filter :
	%node at (0,0)[right=-3mm]{\Large \textopenbullet};
	\node [input, name=input1] {};
    \node [block, name=ret1, right of=input1] (ret1) {$D$};
    \node [block, name=ret2, right of=ret1] (ret2) {$D$};
    \node [right of=ret2] (output) {$y_2$};

    % Joining blocks. 
    % Commands \draw with options like [->] must be written individually
	\draw[draw,->](input1) -- node[name=u] {$u_l$} (ret1);
	\node [sum, below of=u, yshift=-1cm] (sum1) {+};
 	\draw[->](ret1) -- node[name=s1] {$u_{l-1} = s_1$} (ret2);
 	\node [sum, below of=s1, yshift=-1cm] (sum2) {+};
 	\draw[->] (ret2) edge node[name=s2] {$u_{l-2} = s_2$} (output);
	\node [sum, below of=s2, yshift=-1cm] (sum3) {+};
	\node [sum, above of=s2, yshift=0.5cm] (sum4) {+};
 	\draw[->] (u) edge (sum1);
 	\draw[->] (s1) edge (sum2);
 	\draw[->] (s2) edge (sum3);
 	\draw[->] (s2) edge (sum4);
 	\draw[->] (u) |- (sum4);
 	\draw[->] (sum1) edge (sum2);
 	\draw[->] (sum2) edge (sum3);
 	\node[above of=output, yshift=-1.25cm] (y1) {$y_1$};
 	\node[below of=output, yshift=1.25cm] (y3) {$y_3$};
 	\draw[->] (sum4) edge (y1);
	\draw[->] (sum3) edge (y3);

\end{tikzpicture}
\caption{Realization diagram of convolution binary code with generator $\mathbf{g}$}
\label{fig:diag}
\end{figure}

% state update & output function
Let $\s_l = [s_{1, l}, s_{2, l}]^T$ be the state vector at time $l$. It can be immediately seen that the state update function $\SU{\s_l, u_l}$ is
\begin{equation}
	\s_{l+1} = \SU{\s_l, u_l} = \begin{bmatrix} 1 \\ 0 \\ \end{bmatrix} u_l + 
				\begin{bmatrix} 0 & 0 \\ 1 & 0 \\ \end{bmatrix} \s_l
\end{equation}
Let $\mathbf{y}_l = [y_{1, l}, y_{2, l}, y_{3, l}]^T$ be the codeword obtained from input $u_l$ and state $\mathbf{s}_l$. Then the output function $\outf{\s_l, u_l}$ is
\begin{equation}
	\y_l = \outf{\s_l, u_l} = \begin{bmatrix} 1 \\ 0 \\ 1 \\ \end{bmatrix} u_l + 
		\begin{bmatrix} 0 & 1 \\ 0 & 1 \\ 1 & 1 \\ \end{bmatrix} \s_l
\end{equation}

% N(a)

By using the state update function $\SU{\s_l, u_l}$ it is possible to derive the \emph{neighbors set} of each state $\mathbf{a}$, which is the set of couples $(\s_l, u_l) = (\mathbf{b}, u)$ that leads to $\mathbf{a}$:
\begin{equation}
	\mathcal{N}(\mathbf{a}) = \left\{ (\mathbf{b}, u) \middle| \mathbf{a} = \SU{\mathbf{b}, u} \right\} 
\end{equation}
Then
\begin{equation}
	\mathcal{N}(00) = \left\{ ([00], 0), ([01], 0) \right\} 
\end{equation}
\begin{equation*}
	\mathcal{N}(01) = \left\{ ([10], 0), ([11], 0) \right\} 
\end{equation*}
\begin{equation*}
	\mathcal{N}(10) = \left\{ ([00], 1), ([01], 1) \right\} 
\end{equation*}
\begin{equation*}
	\mathcal{N}(01) = \left\{ ([10], 1), ([11], 1) \right\} 
\end{equation*}

% state trans diagr
The state transition diagram is a way to represent the code in term of possible transitions from a certain state and associated output given a certain input. The state transition diagram of $\mathbf{g}$ can be seen in Fig.~\ref{fig:state}.

\begin{figure}
\centering
\begin{tikzpicture}[auto, thick, node distance=4.5cm, ->,>=latex',shorten >=1pt]

	\node [circlenode] (00) {$00$};
	\node [circlenode, above right of=00] (01) {$01$};
	\node [circlenode, below right of=00] (10) {$10$};
	\node [circlenode, above right of=10] (11) {$11$};

	\path[every node]
		(00) edge [loop left] node[left] {$0, [000]$} (00)
			 edge [bend right] node[left] {$1, [101]$} (10)
		(01) edge [bend right] node[left] {$0, [111]$} (00)
			 edge [bend right] node[left] {$1, [010]$} (10)
		(10) edge [bend right] node[right] {$0, [001]$} (01)
			 edge [bend right] node[right] {$1, [100]$} (11)
		(11) edge [loop right] node[right] {$1, [011]$} (11)
			 edge [bend right] node[right] {$0, [110]$} (01);
	
\end{tikzpicture}
\caption{State transition diagram}
\label{fig:state}
\end{figure}

% T
From a slightly modified version of the state transition diagram (in Fig.~\ref{fig:state_t}) it is possible to derive the expression of the transfer function that characterizes the code. The loop on $00$ is removed and this state is split into starting state $\mathbf{0}_s$ and ending state $\mathbf{0}_e$.

\begin{figure}
\centering
\begin{tikzpicture}[auto, thick, node distance=4.5cm, ->,>=latex',shorten >=1pt]
	\node [circlenode] (0s) {$\mathbf{0}_s$};
	\node [circlenode, right of=0s] (10) {$10$};
	\node [circlenode, below right of=10] (11) {$11$};
	\node [circlenode, above right of=11] (01) {$01$};
	\node [circlenode, right of=01] (0e) {$\mathbf{0}_e$};

	\path[every node]
		(0s) edge node[above] {$1, [101]$} node[below] {$x^2wz$} (10)
		(10) edge [bend left] node[above] {$0, [001]$} node[below] {$xz$} (01)
			 edge [bend right] node[left] {$1, [100]$} node[right] {$xwz$} (11)
		(01) edge [bend left] node[above] {$1, [010]$} node[below] {$xwz$} (10)
			 edge node[above] {$0, [111]$} node[below] {$x^3z$} (0e)
		(11) edge [loop below] node[left] {$1, [011]$} node[right] {$x^2wz$} (11)
			 edge [bend right] node[left] {$0, [110]$} node[right] {$x^2z$}(01);
\end{tikzpicture}
\caption{Modified transition diagram}
\label{fig:state_t}
\end{figure}

The on each transition let $x$ be a variable whose exponent is associated with the Hamming weight of the output (i.e. if $||\outf{\s_l, u_l}||_H = d$, then on that edge there will be $x^d$), $w$ be a variable that represents the input weight (thus either with exponent $i$ 0 or 1) and $z$ be associated with the input length $l$ (in this case $l=1$). Then it is possible to label each transition with a $x^dw^iz^l$ triplet of variables and define the function
\begin{equation}
	W_{\mathbf{s}}(x, w, z) = \sum_{d, i, l = 0}^{\infty}W_{d, i, l} x^dw^iz^l
\end{equation}
with the weight $W_{d, i, l}$ given by the number of codeword $\mathbf{y}$ starting from state $00$, ending in state $\mathbf{s}$, with Hamming norm $d$, input weight $i$ and length $l$. 

Let $T(x, w, z) = W_{\mathbf{0}_e} (x, w, z)$ be the transfer function of system, i.e. the function whose terms identify the number of codewords with $d < 2d_{min}$ with a certain input weight $i$ and length $l$. This can be derived by solving the following linear system
\begin{equation}\label{eq:s1}
\begin{cases}
	W_{\mathbf{0}_s} = 1 \\ %TODO definition \\
	W_{01} = W_{11}x^2z + W_{10}xz \\
	W_{10} = x^2wz + W_{01}xwz \\
	W_{11} = W_{11}x^2wz + W_{10}xwz \\
	T = W_{\mathbf{0}_e} = W_{01}x^3z\\
\end{cases}
\end{equation}

\begin{equation}
\begin{cases}
	W_{\mathbf{0}_s} = 1 \\ %TODO definition \\
	W_{01} = W_{10}\frac{x^3wz^2}{1-x^2wz} + W_{10}xz \\
	W_{10} = x^2wz + W_{01}xwz \\
	W_{11} = W_{10}\frac{xwz}{1-x^2wz} \\
	T = W_{\mathbf{0}_e} = W_{01}x^3z\\
\end{cases}
\end{equation}
and the by focusing on $W_{01}$
\begin{equation*}
	W_{01} =  \frac{xz}{1-x^2wz}W_{10}
\end{equation*}
\begin{equation*}
	W_{01}  =  \frac{x^2wz^2}{1-x^2wz}(x + W_{01})
\end{equation*}
\begin{equation*}
	W_{01}\left(\frac{1 - x^2wz - x^2wz^2}{1-x^2wz} \right) = \frac{x^3wz^2}{1-x^2wz}
\end{equation*}
\begin{equation*}
	W_{01} = \frac{x^3wz^2}{1-x^2wz(1+z)}
\end{equation*}
and finally from~\eqref{eq:s1}
\begin{equation}\label{eq:T}
	T(x, w, z) = W_{\mathbf{0}_e}(x, w, z) = \frac{x^6wz^3}{1-x^2wz(1+z)} = x^6wz^3\sum_{k=0}^{\infty} [x^2wz(1+z)]^k
\end{equation}
% Pbit
It is possible to expand $T(x,w,z) = \sum{d, i, l = 0}^{\infty} t_{d, i, l}x^dw^iz^l$ in order to get the coefficients $t_{d,i,l}$ that represent the number of codeword with Hamming weight $d$, associated with an input of weight $i$ and length $l$. From~\eqref{eq:T} I get
\begin{equation}\label{Texp}
	T(x, w, z) = x^6wz^3 + x^8w^2z^4 + x^8w^2z^5 + x^{10}w^3z^5 + 2x^{10}w^3z^6 + x^{10}w^3z^7 + x^6wz^3\sum_{k=4}^{\infty}[x^2wz(1+z)]^k
\end{equation}
It can be seen that $d_{min}=6$ for this code. Since the transfer function $T$ considers only codewords that do not visit more than twice $00$ in their path (i.e. it is only the starting and ending state) it provides meaningful $t_{d, i, l}$ coefficients only for $d < 2d_{min} = 12$. The last infinite sum in~\eqref{Texp} yields terms with $d \ge 2d_{min}$ that cannot be considered in the derivation of $P_{bit}$ expression. The valid $t_{d, i, l}$ coefficients are
\begin{equation*}
	t_{6,1,3} = 1
\end{equation*}
\begin{equation}\label{eq:coeff}
	t_{8,2,4} = t_{8,2,5} = 1
\end{equation}
\begin{equation*}
	t_{10, 3, 5} = t_{10, 3, 7} = 1, \quad  t_{10,3,6} = 2
\end{equation*}
From \cite{erseghe} the bound for $P_{bit}$ is
\begin{equation}
	P_{bit} \le \sum_{d=d_{min}}^n K(d)Q\left(\sqrt{2\frac{E_b}{No}Rd}\right)
\end{equation}
with $K(d) = \frac{1}{k} \sum_{\mathbf{a}\in\mathcal{U}(d)} ||\mathbf{a}||_H$ and $\mathcal{U}_d = \left\{ \mathbf{u} \in \mathcal{U} \middle| \; ||\mathbf{Gu} ||_{H} = d \right\}$. For a convolutional code 
\begin{equation}
	K(d) = \frac{1}{\mu}\sum_{l=0}^{\infty} (\mu + \nu - l) \sum_{i=0}^{\infty} i t_{d,i,l} \quad d < 2d_{min}
\end{equation}
that can be upper bounded by
\begin{equation}
	K(d) \underset{\sim}{<} \sum_{i=0}^{\infty} i \left( \sum_{l=0}^{\infty}t_{d,i,l}\right) \quad d < 2d_{min}
\end{equation}
Then from~\eqref{eq:coeff} 
\begin{equation*}
	K(6) \lesssim 1\cdot t_{6,1,3} = 1
\end{equation*}
\begin{equation}
	K(8) \lesssim 2\cdot (t_{8,2,4} + t_{8,2,5}) = 4
\end{equation}
\begin{equation*}
	K(10) \lesssim 3\cdot (t_{10, 3, 5} + t_{10, 3, 7} + t_{10,3,6}) = 12
\end{equation*}
and eventually the expression for the bit error rate is
\begin{equation}\label{eq:BER_bound}
	P_{bit} \lesssim Q\left(\sqrt{4\frac{E_b}{N_0}} \right) + 4Q\left(\sqrt{\frac{16E_b}{3N_0}} \right) + 12Q\left(\sqrt{\frac{20E_b}{3N_0}} \right)
\end{equation}
given that the rate is $R=1/3$. An approximate expression (but underestimated for low SNR) is
\begin{equation}\label{eq:BER_approx}
	P_{bit} \cong Q\left(\sqrt{4\frac{E_b}{N_0}} \right)
\end{equation}

% plot expected
In Fig.~\ref{fig:BER_theory} there is the plot of expressions~\eqref{eq:BER_bound} and~\eqref{eq:BER_approx}. The nominal coding gain is by definition $\gamma_c = Rd_{min} = 3.01$ dB, while the effective coding gain is the difference in dB between the value of $E_b/N_0$ for which the curves~\eqref{eq:BER_bound} and $Q\left(\sqrt{2\frac{E_b}{N_0}}\right)$ (uncoded $P_{bit}$) reach $P_{bit} = P_{target} = 10^{-5}$. It can be numerically evaluated with MATLAB and for the code under analysis is $\gamma_e = 2.935$ dB. Therefore the difference between nominal and effective coding gain is $\gamma_c - \gamma_e = 0.075$ dB.
\begin{figure}[t]
\centering
\setlength\fheight{0.5\textwidth}
\setlength\fwidth{0.7\textwidth}
% This file was created by matlab2tikz.
%
%The latest updates can be retrieved from
%  http://www.mathworks.com/matlabcentral/fileexchange/22022-matlab2tikz-matlab2tikz
%where you can also make suggestions and rate matlab2tikz.
%
\definecolor{mycolor1}{rgb}{0.00000,0.44700,0.74100}%
\definecolor{mycolor2}{rgb}{0.85000,0.32500,0.09800}%
\definecolor{mycolor3}{rgb}{0.92900,0.69400,0.12500}%
%
\begin{tikzpicture}
\tikzstyle{every node}=[font=\small]
\begin{axis}[%
width=0.951\fwidth,
height=\fheight,
at={(0\fwidth,0\fheight)},
scale only axis,
xmin=-4.00000000000000000000,
xmax=14.00000000000000000000,
xlabel={$\frac{E_b}{N_0}$},
xmajorgrids,
ymode=log,
ymin=0.00000000010000000000,
ymax=1.00000000000000000000,
yminorticks=true,
ylabel={BER},
ymajorgrids,
yminorgrids,
minor grid style={thick}
axis background/.style={fill=white},
legend style={legend cell align=left,align=left,draw=white!15!black}
]
\addplot [color=mycolor1,solid,line width=1.2pt,mark size=3.5pt,mark=diamond,mark options={solid}]
  table[row sep=crcr]{%
-3.00000000000000000000	0.68791077638501962888\\
-2.00000000000000000000	0.43088443164010159503\\
-1.00000000000000000000	0.24474683174340633451\\
0.00000000000000000000	0.12353244966888282663\\
1.00000000000000000000	0.05414607394657723616\\
2.00000000000000000000	0.02010482072296526992\\
3.00000000000000000000	0.00616610874429029954\\
4.00000000000000000000	0.00152328831698934691\\
5.00000000000000000000	0.00029452619928249305\\
6.00000000000000000000	0.00004264874437591205\\
7.00000000000000000000	0.00000428978356831905\\
8.00000000000000000000	0.00000026704862565815\\
9.00000000000000000000	0.00000000881725019549\\
10.00000000000000000000	0.00000000012754633730\\
11.00000000000000000000	0.00000000000064136051\\
12.00000000000000000000	0.00000000000000084550\\
13.00000000000000000000	0.00000000000000000021\\
};
\addlegendentry{\eqref{eq:BER_bound}};

\addplot [color=mycolor2,solid,line width=1.2pt,mark size=1.4pt,mark=square,mark options={solid}]
  table[row sep=crcr]{%
-3.00000000000000000000	0.07840362692723677751\\
-2.00000000000000000000	0.05606898586825975750\\
-1.00000000000000000000	0.03733371276514092091\\
0.00000000000000000000	0.02275013194817921899\\
1.00000000000000000000	0.01241501335411008405\\
2.00000000000000000000	0.00590366586865526555\\
3.00000000000000000000	0.00236347685109398065\\
4.00000000000000000000	0.00076275521709520710\\
5.00000000000000000000	0.00018787219060808341\\
6.00000000000000000000	0.00003296365099183522\\
7.00000000000000000000	0.00000377713219040470\\
8.00000000000000000000	0.00000025333077226887\\
9.00000000000000000000	0.00000000866367213032\\
10.00000000000000000000	0.00000000012698142947\\
11.00000000000000000000	0.00000000000064085510\\
12.00000000000000000000	0.00000000000000084542\\
13.00000000000000000000	0.00000000000000000021\\
};
\addlegendentry{\eqref{eq:BER_approx}};

\addplot [color=mycolor3,solid,line width=1.2pt,mark size=2.0pt,mark=o,mark options={solid}]
  table[row sep=crcr]{%
-3.00000000000000000000	0.15836831880959795216\\
-2.00000000000000000000	0.13064448852282922742\\
-1.00000000000000000000	0.10375909595340632174\\
0.00000000000000000000	0.07864960352514256681\\
1.00000000000000000000	0.05628195197654145554\\
2.00000000000000000000	0.03750612835892597891\\
3.00000000000000000000	0.02287840756108534129\\
4.00000000000000000000	0.01250081804073756644\\
5.00000000000000000000	0.00595386714777865981\\
6.00000000000000000000	0.00238829078093280751\\
7.00000000000000000000	0.00077267481537844445\\
8.00000000000000000000	0.00019090777407599314\\
9.00000000000000000000	0.00003362722841961758\\
10.00000000000000000000	0.00000387210821552205\\
11.00000000000000000000	0.00000026130679535752\\
12.00000000000000000000	0.00000000900601035063\\
13.00000000000000000000	0.00000000013329310175\\
};
\addlegendentry{Uncoded};

\end{axis}
\end{tikzpicture}%
\caption{BER from expressions~\eqref{eq:BER_bound} and~\eqref{eq:BER_approx}}
\label{fig:BER_theory}
\end{figure}

% plot comparison with simulation
In order to evaluate through simulation the performance of the convolutional code with generator polynomial $\mathbf{g}$ an encoder and a Viterbi decoder were implemented in C using MATLAB's MEX functions. The channel implementation instead is left to a sum between vectors in MATLAB, since the \texttt{randn} function of MATLAB generates random Gaussian noise with a very fast algorithm (Ziggurat algorithm by Masaglia, \cite{moler}).
In Fig.~\ref{fig:BER_1} there is the comparison between the simulated results and the theoretical bounds.
In Fig.~\ref{fig:BER_2} there is the comparison between a Viterbi that backtracks on the complete trellis and a windowed version for 3 different window size.

% give params and implementation details

\begin{figure}[t]
\centering
\setlength\fheight{0.5\textwidth}
\setlength\fwidth{0.7\textwidth}
% This file was created by matlab2tikz.
%
%The latest updates can be retrieved from
%  http://www.mathworks.com/matlabcentral/fileexchange/22022-matlab2tikz-matlab2tikz
%where you can also make suggestions and rate matlab2tikz.
%
\definecolor{mycolor1}{rgb}{0.00000,0.44700,0.74100}%
\definecolor{mycolor2}{rgb}{0.85000,0.32500,0.09800}%
\definecolor{mycolor3}{rgb}{0.92900,0.69400,0.12500}%
\definecolor{mycolor4}{rgb}{0.49400,0.18400,0.55600}%
\definecolor{mycolor5}{rgb}{0.46600,0.67400,0.18800}%
%
\begin{tikzpicture}

\begin{axis}[%
width=0.951\fwidth,
height=\fheight,
at={(0\fwidth,0\fheight)},
scale only axis,
xmin=-4.00000000000000000000,
xmax=8.00000000000000000000,
xlabel={$\frac{E_b}{N_0}$},
xmajorgrids,
ymode=log,
ymin=0.00000010000000000000,
ymax=1.00000000000000000000,
yminorticks=true,
ylabel={BER},
ymajorgrids,
yminorgrids,
axis background/.style={fill=white},
legend style={at={(0.01,0.01)},legend cell align=left,align=left,anchor=south west,draw=white!15!black}
]
\addplot [color=mycolor1,solid,line width=1.2pt,mark size=2.0pt,mark=x,mark options={solid}]
  table[row sep=crcr]{%
-3.00000000000000000000	0.19189999999999998725\\
-2.00000000000000000000	0.14430000000000001159\\
-1.00000000000000000000	0.10150000000000000688\\
0.00000000000000000000	0.06444999999999999341\\
1.00000000000000000000	0.03366666666666666419\\
2.00000000000000000000	0.01408749999999999933\\
3.00000000000000000000	0.00458636363636363668\\
4.00000000000000000000	0.00139999999999999999\\
5.00000000000000000000	0.00028818443804034583\\
6.00000000000000000000	0.00004158703780639801\\
7.00000000000000000000	0.00000416000000000000\\
8.00000000000000000000	0.00000022000000000000\\
};
\addlegendentry{Simulation, random input};

\addplot [color=mycolor2,solid,line width=1.2pt,mark size=2.0pt,mark=o,mark options={solid}]
  table[row sep=crcr]{%
-3.00000000000000000000	0.20719999999999999529\\
-2.00000000000000000000	0.16209999999999999409\\
-1.00000000000000000000	0.11250000000000000278\\
0.00000000000000000000	0.06629999999999999782\\
1.00000000000000000000	0.03409999999999999837\\
2.00000000000000000000	0.01545714285714285680\\
3.00000000000000000000	0.00470000000000000018\\
4.00000000000000000000	0.00131168831168831175\\
5.00000000000000000000	0.00028356940509915014\\
6.00000000000000000000	0.00004072416598860863\\
7.00000000000000000000	0.00000378000000000000\\
8.00000000000000000000	0.00000023000000000000\\
};
\addlegendentry{Simulation, zero input};

\addplot [color=mycolor3,solid,line width=1.2pt,mark size=3.5pt,mark=diamond,mark options={solid}]
  table[row sep=crcr]{%
-3.00000000000000000000	0.68791077638501962888\\
-2.00000000000000000000	0.43088443164010159503\\
-1.00000000000000000000	0.24474683174340633451\\
0.00000000000000000000	0.12353244966888282663\\
1.00000000000000000000	0.05414607394657723616\\
2.00000000000000000000	0.02010482072296526992\\
3.00000000000000000000	0.00616610874429029954\\
4.00000000000000000000	0.00152328831698934691\\
5.00000000000000000000	0.00029452619928249305\\
6.00000000000000000000	0.00004264874437591205\\
7.00000000000000000000	0.00000428978356831905\\
8.00000000000000000000	0.00000026704862565815\\
};
\addlegendentry{\eqref{eq:BER_bound}};

\addplot [color=mycolor4,solid,line width=1.2pt,mark size=1.4pt,mark=square,mark options={solid}]
  table[row sep=crcr]{%
-3.00000000000000000000	0.07840362692723677751\\
-2.00000000000000000000	0.05606898586825975750\\
-1.00000000000000000000	0.03733371276514092091\\
0.00000000000000000000	0.02275013194817921899\\
1.00000000000000000000	0.01241501335411008405\\
2.00000000000000000000	0.00590366586865526555\\
3.00000000000000000000	0.00236347685109398065\\
4.00000000000000000000	0.00076275521709520710\\
5.00000000000000000000	0.00018787219060808341\\
6.00000000000000000000	0.00003296365099183522\\
7.00000000000000000000	0.00000377713219040470\\
8.00000000000000000000	0.00000025333077226887\\
};
\addlegendentry{\eqref{eq:BER_approx}};

\addplot [color=mycolor5,solid,line width=1.2pt,mark size=2.0pt,mark=o,mark options={solid}]
  table[row sep=crcr]{%
-3.00000000000000000000	0.15836831880959795216\\
-2.00000000000000000000	0.13064448852282922742\\
-1.00000000000000000000	0.10375909595340632174\\
0.00000000000000000000	0.07864960352514256681\\
1.00000000000000000000	0.05628195197654145554\\
2.00000000000000000000	0.03750612835892597891\\
3.00000000000000000000	0.02287840756108534129\\
4.00000000000000000000	0.01250081804073756644\\
5.00000000000000000000	0.00595386714777865981\\
6.00000000000000000000	0.00238829078093280751\\
7.00000000000000000000	0.00077267481537844445\\
8.00000000000000000000	0.00019090777407599314\\
};
\addlegendentry{Uncoded};

\end{axis}
\end{tikzpicture}%
\caption{BER from expressions~\eqref{eq:BER_bound} and~\eqref{eq:BER_approx} vs simulated BER}
\label{fig:BER_1}
\end{figure}



%%%%%%%%%%%%%%%%%%%%%%%%%%%%%%%%%%%%%%%%%
%%%%%%%%%%%%%% PROBLEM 2 %%%%%%%%%%%%%%%%
%%%%%%%%%%%%%%%%%%%%%%%%%%%%%%%%%%%%%%%%%

\section*{Exercise 2}

\begin{thebibliography}{10}

\end{thebibliography}

\end{document}
