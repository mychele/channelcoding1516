\documentclass[pdf]
          {beamer}
\mode<presentation>{}
\usetheme{Pittsburgh}
\usecolortheme{beaver}
\usefonttheme{professionalfonts}
% Various useful packages and commands from different sources

%\usepackage[applemac]{inputenc}
\usepackage[english]{babel}
\usepackage[T1]{fontenc}
\usepackage{cite, url,color} % Citation numbers being automatically sorted and properly "compressed/ranged".


% Compact lists
%\usepackage{enumitem}
\usepackage{booktabs}
\usepackage{fancyvrb}

% Tikz
\usepackage{tikz}
\usetikzlibrary{automata,positioning,chains,shapes,arrows}
\usepackage{pgfplots}
\usetikzlibrary{plotmarks}
\newlength\fheight
\newlength\fwidth
\pgfplotsset{compat=newest}
\pgfplotsset{plot coordinates/math parser=false}
\usetikzlibrary{calc}

\usepackage{array}
\usepackage{kbordermatrix}
% http://www.ctan.org/tex-archive/macros/latex/required/tools/
\usepackage{mdwmath}
\usepackage{mdwtab}
%mdwtab.sty	-- A complete ground-up rewrite of LaTeX's `tabular' and  `array' environments.  Has lots of advantages over
%		   the standard version, and over the version in `array.sty'.
% *** SUBFIGURE PACKAGES ***
% \usepackage[tight,footnotesize]{subfigure}
\usepackage{subfig}
%\usepackage{indentfirst}


%\setlength\parindent{0pt}
\linespread{1}

\usepackage{mathtools}
\DeclarePairedDelimiter{\ceil}{\lceil}{\rceil}
\DeclarePairedDelimiter{\floor}{\lfloor}{\rfloor}
\DeclareMathOperator*{\argmax}{arg\,max}
\DeclareMathOperator*{\argmin}{arg\,min}
\newcommand{\M} {\mathtt{M}}
\newcommand{\dB} {\mathrm{dB}}
\newcommand{\tr} {\mathrm{tr}}
\newcommand{\lmod}[1] {_{\,\mathrm{mod}\,#1}}
\newcommand{\outf}[1] {\mathcal{O}(#1)}
\newcommand{\SU}[1] {\mathcal{S}(#1)}
\newcommand{\s} {\mathbf{s}}
\newcommand{\y} {\mathbf{y}}


\graphicspath{ {figures/} }
\setcounter{MaxMatrixCols}{20}

%% preamble
\title{Implementation and performance evaluation of ITU-T G.975.1 LDPC binary code}
\subtitle{Channel Coding 15/16 Final Project}
\author{Michele Polese}
\begin{document}

% For tikz
% Definition of blocks:
\tikzstyle{block} = [draw, rectangle, 
    minimum height=1em, minimum width=1em]
\tikzstyle{circlenode} = [draw, circle, minimum height=1em, minimum width=3em]
\tikzstyle{trellisnode} = [draw, circle, minimum height=2em, minimum width=2em]
\tikzstyle{sum} = [draw, circle, node distance=1cm]
\tikzstyle{input} = [coordinate]
\tikzstyle{output} = [coordinate]
\tikzstyle{pinstyle} = [pin edge={to-,thin,black}]

%% title frame
\begin{frame}
    \titlepage
\end{frame}
%% normal frame 
\begin{frame}{Outline}
    \begin{itemize}
    	\item Field of application: DWDM submarine systems
 		\item LDPC Encoder: encoding matrix and implementation
 		\item Message passing decoder
 		\item LDPC Decoder C++ implementation: the flexibility of OOP
 		\item Performance evaluation
 		\item Conclusions
	\end{itemize}
\end{frame}

\begin{frame}{DWDM Submarine Optical Systems}
	\begin{columns}
		\column{0.6\textwidth}
			\begin{figure}
				\centering
				\includegraphics[width = 1.1\textwidth]{dwdm}
			\end{figure}
		\column{0.4\textwidth}
			\begin{itemize}
				\item DWDM interfaces with different optical transport networks
				\item The channel can be modeled as a Gaussian channel
			\end{itemize}
	\end{columns}
\end{frame}

\begin{frame}{ITU-T G.975.1}
	\begin{center}
	Forward error correction for high bit-rate DWDM submarine systems 
	\end{center}
	\begin{itemize}
		\item \textit{Super FEC} schemes for coding in submarine optical systems
		\item More robust than ITU-T G.975 FEC - RS (255, 239)
		\item Concatenate RS or BCH with different options
		\item LDPC (32640, 30592)
	\end{itemize}
	
\end{frame}

\begin{frame}{LDPC (32640, 30592)}
	\begin{itemize}
		\item Information word with $K=30592$, it fits a RS (255,239) frame
		\item High coding rate $R = \frac{K}{N} = 0.9374$
		\item Spectral efficiency $\rho = \frac{R \log_2(M)}{BT} = 2R = 1.8748$
		\item Hardware implementation suitable for application with 10G and 40G fibers
	\end{itemize}
	
\end{frame}

\begin{frame}{Encoding procedure 1/2}
	\begin{itemize}
		\item The information bit are placed in a $112\times 293$ matrix $\mathbf{S}$
		\item Bit $j$, $j\in[1, 30592]$, is inserted in position $(r, 293r + 292 - q)$ with
		\begin{equation*}
			r = \left\lfloor\dfrac{j}{293}\right\rfloor
		\end{equation*}
		\begin{equation*}
			q = j + 172
		\end{equation*}
		\item Entries in $(0, 292-d), d \in [0, 172]$ are set to 0 and never transmitted
		\item 7 slopes $s_i, \, i \in \{1 \dots 7\}$ are chosen
		\item For each slope $s_i$ 293 lines are defined by
		\begin{equation*}
			(a, b) | b = (s_ia + c)\mod_{293}, \quad c \in [0, 292]
		\end{equation*}
	\end{itemize}
\end{frame}

\begin{frame}{Encoding procedure 2/2}
	\begin{itemize}
		\item A total of 2051 lines are defined
		\item The sum (modulo 2) of the bits in each line must be 0
		\item The parity check equations define a system of 2051 equations in 2051 unknowns
		\item 6 parity check bit are redundant, and removed from the linear system, as well as the last equation ($c=292$) for the first 6 slopes
		\item This system can be written as 
		\begin{equation*}
			\mathbf{H} \mathbf{c} = \mathbf{0}
		\end{equation*}
	\end{itemize}
\end{frame}

\begin{frame}{Matrix $\mathbf{H}$}
	\begin{equation*}
		\mathbf{H} = 
			\kbordermatrix{%
					& 105\times293 & 7\times293 - 6 \\
				2045 & \mathbf{M} & \mathbf{N}
			}
	\end{equation*}
	Each row $i$ of $\mathbf{H}$ is defined by a valid couple $(s_i, c_i)$, and column $j$ corresponds to bit $(\floor{j/293}, j\mod_{293})$ in matrix $\mathbf{S}$. Then
	\begin{equation*}
		h_{i, j} = \begin{cases}
			1, &\text{ if }j\mod{293} == s_i\floor{\frac{j}{293}} + c_i\\
			0, &\text{ otherwise}\\
		\end{cases}
	\end{equation*}
	Given the line $(s_i, c_i)$, a column of $\mathbf{M}$ contains a 1 if the \textit{information} bit in the related position belongs to the line, $\mathbf{N}$ if a \textit{parity check} bit belongs to the line.
\end{frame}

\begin{frame}{From $\mathbf{H}$ to $\mathbf{G}$}
	$\mathbf{H}$ is transformed to compute the encoding matrix $\mathbf{G}$
	\begin{equation*}
		\mathbf{H}_{to inv} = 
			\kbordermatrix{%
					 & 2045 	  & 30765 \\
				2045 & \mathbf{N} & \mathbf{M}
			}
	\end{equation*}
	\begin{equation*}
		\mathbf{H}_{to inv} \; | \; \mathbf{I}_{2045} = 
			\kbordermatrix{%
					 & 2045 	  & 30765  & 2045\\
				2045 & \mathbf{N} & \mathbf{M} & \mathbf{I}_{2045}
			}
	\end{equation*}
	\small{Gauss elimination is applied to bring $\mathbf{H}_{toinv} \; | \; \mathbf{I}_{2045}$ in a row echelon form, then Jordan algorithm is used to isolate an identity matrix in first 2045 columns. The result is}
	\begin{equation*}
			\kbordermatrix{%
					 & 2045 	  & 30765  & 2045\\
				2045 & \mathbf{I}_{2045} & \mathbf{N}^{-1}\mathbf{M} & \mathbf{N}^{-1}
			}
	\end{equation*}
	\small{and finally}
	\begin{equation*}
			\mathbf{\tilde{H}} = 
			\kbordermatrix{%
					 & 30765  & 2045\\
				2045 & \mathbf{N}^{-1}\mathbf{M} & \mathbf{I}_{2045} 
			}
	\end{equation*}
\end{frame}
\renewcommand{\arraystretch}{1.5}
\begin{frame}{Matrix $\mathbf{G}$}
	Matrix $\mathbf{G}$ is obtained as 
	\begin{equation*}
			\mathbf{G} = 
			\kbordermatrix{%
					 & 30765 \\
				30765 & \mathbf{I}_{30765} \\ 
				2045 &  \mathbf{N}^{-1}\mathbf{M}  
			}
	\end{equation*}
	For the Gauss elimination NTL\footnote{\url{http://www.shoup.net/ntl/}} library is used. Then each row of matrix $\mathbf{K} = \mathbf{N}^{-1}\mathbf{M}$ is saved into a \texttt{std::bitset} and stored to file. 
\end{frame}

\begin{frame}{Encoder}
\begin{itemize}
	\item The encoder is implemented as a \texttt{C++} object. Upon initialization, matrix $\mathbf{K}$ is read from file and loaded in memory. 
	\item Both infoword and codeword are \texttt{std::bitset}
	\item Encoding is performed by filling the first 30592 bit of the codeword with the information word, and by computing the 2045 parity check bit with an \texttt{and} operation between the infoword and the corresponding row of matrix $\mathbf{K}$
	\item Three bit set to 0 are inserted between the information word and the parity check bits
\end{itemize}
\end{frame}

\begin{frame}{Message Passing Decoder}
	\begin{columns}
	\column{0.6\framewidth}
		\begin{figure}
		\centering
		\begin{tikzpicture}[auto, thick, node distance=1 cm]
			\node [block] (u0) {};
			\node [block, below of=u0] (u1) {};
			\node [block, below of=u1] (u2) {};
			\node [block, below of=u2] (u3) {};
			\node [block, below of=u3, yshift=-1cm] (ck) {};
			\draw[densely dotted](u3) -- node[name=u] {} (ck);

			\node [block, right of=u0] (v0) {=};
			\node [block, right of=u1] (v1) {=};
			\node [block, right of=u2] (v2) {=};
			\node [block, right of=u3] (v3) {=};
			\node [block, right of=ck] (vk) {=};

			\node [draw, rectangle, minimum height=16em, minimum width=3em, right of=v2, xshift=1cm, yshift=-0.5cm] (perm) {\huge{$\mathbf{\pi}$}};

			\node [block, right of=perm, xshift=1.5cm] (c1) {+};
			\node [block, above of=c1, yshift=1cm] (c0) {+};
			\node [block, below of=c1, yshift=-1cm] (c2) {+};

			\draw [solid](u0) -- node[name=l0] {} (v0);
			\draw [solid](u1) -- node {} (v1);
			\draw [solid](u2) -- node {} (v2);
			\draw [solid](u3) -- node {} (v3);
			\draw [solid](ck) -- node[name=lk] {} (vk);

			\foreach \y in {2.1,2.5,2.9}{
	    		\draw [-] (v0) -- ($(perm.west) + (0,\y)$);
	  		}
	  		\foreach \y in {1.9,1.5,1.1}{
	    		\draw [-] (v1) -- ($(perm.west) + (0,\y)$);
	  		}
			\foreach \y in {0.9,0.5,0.1}{
	    		\draw [-] (v2) -- ($(perm.west) + (0,\y)$);
	  		}
			\foreach \y in {-0.9,-0.5,-0.1}{
	    		\draw [-] (v3) -- ($(perm.west) + (0,\y)$);
	  		}
	  		\foreach \y in {-2.9,-2.5,-2.1}{
	    		\draw [-] (vk) -- ($(perm.west) + (0,\y)$);
	  		}

	  		\foreach \y in {1.1, 1.5, 2, 2.5, 2.9}{
	    		\draw [-] (c0) -- ($(perm.east) + (0,\y)$);
	  		}
	  		\foreach \y in {-0.9, -0.5, 0, 0.5, 0.9}{
	    		\draw [-] (c1) -- ($(perm.east) + (0,\y)$);
	  		}
			\foreach \y in {-1.1, -1.5, -2, -2.5, -2.9}{
	    		\draw [-] (c2) -- ($(perm.east) + (0,\y)$);
	  		}

	  		\node[above of=l0, xshift=0.5cm, yshift=-0.5cm] (vn) {Variable nodes};
	  		\node[right of=vn, xshift=3.5cm] (cn) {Check nodes};
	  		\node[below of=lk, xshift=-0.3cm, yshift=0.4cm] (ln) {Leaf nodes};


		\end{tikzpicture}
		\caption{Factor graph for LDPC decoding}
		\label{fig:mp}
		\end{figure}
	\column{0.4\framewidth}
	\begin{itemize}
		\item The decoder is based on this factor graph.
		\item Decoding is performed in the LLR domain.
	\end{itemize}
	\end{columns}
\end{frame}

\begin{frame}{LLR and Leaf nodes messages}
	The LDPC code under analysis is a binary code. Therefore the LLR associated to message $\mu$ is expressed as
	$$
		LLR_{\mu} = \ln \left( \frac{\mu(0)}{\mu(1)} \right)
	$$
	Leaf nodes are initialized with received values, and under the hypothesis of equally probable input symbols the LLR are
	$$
		LLR_{g_l \rightarrow c_l} = \ln \left( \frac{\frac{1}{\sqrt{2\pi\sigma_w^2}} e^{-\frac{1}{2\sigma_w^2}(r_l+1)}}{\frac{1}{\sqrt{2\pi\sigma_w^2}}e^{-\frac{1}{2\sigma_w^2}(r_l-1)}} \right) = -\frac{2r_l}{\sigma_w^2}
	$$
\end{frame}

\begin{frame}{Variable Node}
	A variable node represents a delta function, therefore the LLR on each branch is
	$$
		LLR_{=\rightarrow j} = \sum_{i \ne j} LLR_{i \rightarrow =}
	$$
	This LDPC code has variable nodes with 7 branches connected to check nodes, with the exception of variables figuring in linearly dependent parity check equations, which have 6 outgoing branches.
	\begin{figure}
	\centering
	\begin{tikzpicture}[auto, thick, node distance=1 cm]
		\node [block] (vn) {=};
		\foreach \y in {-0.5,0,0.5}{%
   			\draw [-] (vn) -- ($(vn.east) + (1,\y)$);
  		}
  		\draw [-] (vn) -- ($(vn.west) + (-1,0)$);
		\draw [gray, <-] ($(vn.center) + (0.8,-0.7)$) arc [radius=2, start angle=-20, end angle= 20] node[name=countline] {};	
		\node [above of=countline, yshift=-0.7cm] (ul) {7};

	\end{tikzpicture}
	\caption{Variable Node}
	\label{fig:vn}
	\end{figure}
\end{frame}

\begin{frame}{Check Node}
	Each check node is connected to 112 variable nodes, and there are 2045 check nodes. The LLR of outgoing branch $j$ is given by
	$$
		LLR_{+\rightarrow j} = \tilde{\Phi} \left( \sum_{i \ne j} \tilde{\Phi}\left( |LLR_{i \rightarrow +}|\right) \right) \Pi_{i \ne j} \mbox{sign} \left( LLR_{i \rightarrow +} \right)
	$$

	\begin{figure}
	\centering
	\begin{tikzpicture}[auto, thick, node distance=1 cm]
		\node [block] (cn) {+};
		\foreach \y in {-0.1,-0.2,-0.3,-0.4,-0.5,-1,0,0.1,0.2,0.3,0.4,0.5,1}{%
   			\draw [-] (cn) -- ($(cn.east) + (1,\y)$);
  		}
		\draw [gray, <-] ($(cn.center) + (0.8,-0.8)$) arc [radius=2, start angle=-25, end angle= 25] node[name=countline] {};	
		\node [above of=countline, yshift=-0.7cm] (ul) {112};

	\end{tikzpicture}
	\caption{Check Node}
	\label{fig:vn}
	\end{figure}
\end{frame}


\begin{frame}{Check Node}
	\begin{columns}
	\column{0.5\textwidth}
	The function $\tilde{\Phi} (x) $ is given by
	$$
		\tilde{\Phi} = - \ln \left( \tanh \left(\frac{1}{2} x\right)\right)
	$$
	\column{0.5\textwidth}
	\begin{figure}[t]
		\centering
		\setlength\fheight{0.5\textwidth}
		\setlength\fwidth{0.5\textwidth}
		% This file was created by matlab2tikz.
%
%The latest updates can be retrieved from
%  http://www.mathworks.com/matlabcentral/fileexchange/22022-matlab2tikz-matlab2tikz
%where you can also make suggestions and rate matlab2tikz.
%
\definecolor{mycolor1}{rgb}{0.00000,0.44700,0.74100}%
%
\begin{tikzpicture}

\begin{axis}[%
width=0.951\fwidth,
height=\fheight,
at={(0\fwidth,0\fheight)},
scale only axis,
unbounded coords=jump,
xmin=-0.1,
xmax=5.5,
xlabel={$\scriptstyle x$},
xmajorgrids,
ymin=-0.1,
ymax=5.5,
ylabel={{$\scriptstyle \tilde{\Phi}(x)$}},
ymajorgrids,
axis background/.style={fill=white}
]
\addplot [color=mycolor1,solid,line width=1.2pt,forget plot]
  table[row sep=crcr]{%
0	inf\\
0.005	5.99146663043828\\
0.01	5.29832569983276\\
0.015	4.89287100819378\\
0.02	4.60520351854367\\
0.025	4.38207871610843\\
0.03	4.19978007394268\\
0.035	4.04565647409192\\
0.04	3.91215632631843\\
0.045	3.79440869984101\\
0.05	3.68908775707066\\
0.055	3.5938213131702\\
0.06	3.50685783433592\\
0.065	3.42686718623167\\
0.07	3.35281543415097\\
0.075	3.28388294225797\\
0.08	3.21940895917996\\
0.085	3.15885303276087\\
0.09	3.1017674704558\\
0.095	3.04777725558535\\
0.1	2.99656512111766\\
0.105	2.947860268971\\
0.11	2.90142971597252\\
0.115	2.85707156509039\\
0.12	2.81460970977925\\
0.125	2.77388962008037\\
0.13	2.73477495568512\\
0.135	2.69714481854819\\
0.14	2.66089150539072\\
0.145	2.62591865476842\\
0.15	2.59213970839391\\
0.155	2.55947662484966\\
0.16	2.52785879758552\\
0.165	2.49722213946396\\
0.17	2.46750830400634\\
0.175	2.43866401955326\\
0.18	2.41064051724537\\
0.185	2.38339303739464\\
0.19	2.35688040169795\\
0.195	2.33106464102742\\
0.2	2.30591067035211\\
0.205	2.28138600380594\\
0.21	2.2574605040954\\
0.215	2.23410616139737\\
0.22	2.21129689767827\\
0.225	2.18900839300558\\
0.23	2.16721793095106\\
0.235	2.14590426062122\\
0.24	2.12504747321445\\
0.245	2.10462889130705\\
0.25	2.08463096932488\\
0.255	2.0650372038714\\
0.26	2.04583205276348\\
0.265	2.02700086177955\\
0.27	2.00852979825499\\
0.275	1.99040579077083\\
0.28	1.97261647427677\\
0.285	1.95515014007146\\
0.29	1.937995690133\\
0.295	1.9211425953534\\
0.3	1.90458085728337\\
0.305	1.88830097303908\\
0.31	1.87229390306248\\
0.315	1.8565510414612\\
0.32	1.84106418868411\\
0.325	1.82582552631548\\
0.33	1.81082759379335\\
0.335	1.79606326687875\\
0.34	1.78152573771993\\
0.345	1.76720849637194\\
0.35	1.75310531364594\\
0.355	1.73921022517495\\
0.36	1.72551751659409\\
0.365	1.71202170974292\\
0.37	1.6987175498066\\
0.375	1.68559999332003\\
0.38	1.67266419696666\\
0.385	1.65990550710935\\
0.39	1.64731944999683\\
0.395	1.63490172259397\\
0.4	1.62264818398888\\
0.405	1.61055484733373\\
0.41	1.59861787228011\\
0.415	1.58683355787288\\
0.42	1.57519833586958\\
0.425	1.56370876445524\\
0.43	1.55236152232471\\
0.435	1.54115340310716\\
0.44	1.53008131010919\\
0.445	1.519142251355\\
0.45	1.50833333490363\\
0.455	1.49765176442504\\
0.46	1.48709483501787\\
0.465	1.47665992925338\\
0.47	1.46634451343089\\
0.475	1.4561461340315\\
0.48	1.44606241435734\\
0.485	1.43609105134517\\
0.49	1.42622981254322\\
0.495	1.41647653324167\\
0.5	1.4068291137473\\
0.505	1.39728551679367\\
0.51	1.38784376507908\\
0.515	1.37850193892447\\
0.52	1.36925817404462\\
0.525	1.36011065942593\\
0.53	1.35105763530485\\
0.535	1.34209739124121\\
0.54	1.33322826428125\\
0.545	1.32444863720518\\
0.55	1.31575693685497\\
0.555	1.30715163253765\\
0.56	1.29863123450038\\
0.565	1.29019429247318\\
0.57	1.281839394276\\
0.575	1.2735651644865\\
0.58	1.2653702631657\\
0.585	1.25725338463813\\
0.59	1.24921325632409\\
0.595	1.24124863762111\\
0.6	1.23335831883221\\
0.605	1.22554112013872\\
0.61	1.21779589061534\\
0.615	1.21012150728543\\
0.62	1.20251687421455\\
0.625	1.19498092164045\\
0.63	1.18751260513771\\
0.635	1.18011090481546\\
0.64	1.1727748245465\\
0.645	1.1655033912265\\
0.65	1.15829565406184\\
0.655	1.15115068388461\\
0.66	1.14406757249386\\
0.665	1.13704543202161\\
0.67	1.13008339432262\\
0.675	1.1231806103869\\
0.68	1.11633624977393\\
0.685	1.10954950006753\\
0.69	1.10281956635068\\
0.695	1.09614567069924\\
0.7	1.08952705169384\\
0.705	1.08296296394921\\
0.71	1.07645267766007\\
0.715	1.06999547816302\\
0.72	1.06359066551374\\
0.725	1.05723755407875\\
0.73	1.05093547214129\\
0.735	1.0446837615206\\
0.74	1.0384817772042\\
0.745	1.03232888699248\\
0.75	1.02622447115525\\
0.755	1.0201679220997\\
0.76	1.01415864404928\\
0.765	1.00819605273324\\
0.77	1.00227957508614\\
0.775	0.996408648957265\\
0.78	0.990582722829239\\
0.785	0.98480125554576\\
0.79	0.979063716047933\\
0.795	0.973369583118943\\
0.8	0.967718345136753\\
0.805	0.962109499834514\\
0.81	0.956542554068402\\
0.815	0.951017023592616\\
0.82	0.945532432841266\\
0.825	0.940088314716903\\
0.83	0.934684210385452\\
0.835	0.929319669077309\\
0.84	0.923994247894398\\
0.845	0.918707511622951\\
0.85	0.913459032551837\\
0.855	0.908248390296217\\
0.86	0.903075171626361\\
0.865	0.897938970301433\\
0.87	0.892839386908082\\
0.875	0.887776028703658\\
0.88	0.882748509463926\\
0.885	0.87775644933508\\
0.89	0.872799474689958\\
0.895	0.867877217988283\\
0.9	0.86298931764081\\
0.905	0.85813541787725\\
0.91	0.853315168617839\\
0.915	0.84852822534844\\
0.92	0.843774248999051\\
0.925	0.839052905825626\\
0.93	0.834363867295085\\
0.935	0.829706809973416\\
0.94	0.825081415416772\\
0.945	0.820487370065472\\
0.95	0.815924365140799\\
0.955	0.811392096544519\\
0.96	0.806890264761044\\
0.965	0.802418574762127\\
0.97	0.797976735914042\\
0.975	0.793564461887157\\
0.98	0.789181470567826\\
0.985	0.784827483972532\\
0.99	0.780502228164219\\
0.995	0.776205433170729\\
1	0.771936832905305\\
1.005	0.767696165089071\\
1.01	0.763483171175458\\
1.015	0.759297596276495\\
1.02	0.755139189090923\\
1.025	0.751007701834074\\
1.03	0.746902890169465\\
1.035	0.742824513142056\\
1.04	0.738772333113125\\
1.045	0.734746115696712\\
1.05	0.730745629697586\\
1.055	0.726770647050698\\
1.06	0.722820942762068\\
1.065	0.718896294851072\\
1.07	0.714996484294093\\
1.075	0.711121294969483\\
1.08	0.707270513603818\\
1.085	0.703443929719399\\
1.09	0.699641335582965\\
1.095	0.695862526155589\\
1.1	0.692107299043721\\
1.105	0.68837545445135\\
1.11	0.684666795133249\\
1.115	0.680981126349284\\
1.12	0.677318255819743\\
1.125	0.673677993681674\\
1.13	0.670060152446193\\
1.135	0.666464546956745\\
1.14	0.662890994348286\\
1.145	0.659339314007363\\
1.15	0.655809327533073\\
1.155	0.652300858698875\\
1.16	0.648813733415231\\
1.165	0.645347779693055\\
1.17	0.641902827607957\\
1.175	0.638478709265246\\
1.18	0.635075258765693\\
1.185	0.631692312172009\\
1.19	0.628329707476051\\
1.195	0.624987284566707\\
1.2	0.621664885198465\\
1.205	0.618362352960638\\
1.21	0.615079533247231\\
1.215	0.611816273227436\\
1.22	0.608572421816738\\
1.225	0.605347829648617\\
1.23	0.602142349046828\\
1.235	0.598955833998253\\
1.24	0.595788140126307\\
1.245	0.592639124664873\\
1.25	0.589508646432784\\
1.255	0.586396565808797\\
1.26	0.583302744707084\\
1.265	0.580227046553211\\
1.27	0.577169336260587\\
1.275	0.574129480207391\\
1.28	0.571107346213944\\
1.285	0.56810280352054\\
1.29	0.565115722765695\\
1.295	0.562145975964837\\
1.3	0.559193436489397\\
1.305	0.556257979046318\\
1.31	0.553339479657947\\
1.315	0.550437815642326\\
1.32	0.54755286559385\\
1.325	0.544684509364305\\
1.33	0.541832628044255\\
1.335	0.538997103944796\\
1.34	0.53617782057964\\
1.345	0.533374662647545\\
1.35	0.530587516015075\\
1.355	0.527816267699676\\
1.36	0.525060805853073\\
1.365	0.522321019744974\\
1.37	0.519596799747069\\
1.375	0.51688803731733\\
1.38	0.514194624984603\\
1.385	0.511516456333466\\
1.39	0.508853425989383\\
1.395	0.50620542960411\\
1.4	0.503572363841378\\
1.405	0.500954126362824\\
1.41	0.498350615814182\\
1.415	0.495761731811717\\
1.42	0.493187374928905\\
1.425	0.490627446683342\\
1.43	0.488081849523895\\
1.435	0.485550486818071\\
1.44	0.483033262839616\\
1.445	0.480530082756323\\
1.45	0.478040852618059\\
1.455	0.475565479345\\
1.46	0.473103870716065\\
1.465	0.470655935357557\\
1.47	0.468221582731997\\
1.475	0.465800723127146\\
1.48	0.463393267645218\\
1.485	0.460999128192281\\
1.49	0.458618217467831\\
1.495	0.456250448954544\\
1.5	0.453895736908206\\
1.505	0.45155399634781\\
1.51	0.449225143045813\\
1.515	0.446909093518564\\
1.52	0.444605765016892\\
1.525	0.442315075516839\\
1.53	0.440036943710561\\
1.535	0.437771288997369\\
1.54	0.435518031474918\\
1.545	0.433277091930548\\
1.55	0.431048391832758\\
1.555	0.428831853322821\\
1.56	0.426627399206537\\
1.565	0.424434952946119\\
1.57	0.422254438652207\\
1.575	0.420085781076011\\
1.58	0.417928905601588\\
1.585	0.415783738238225\\
1.59	0.41365020561296\\
1.595	0.411528234963213\\
1.6	0.409417754129536\\
1.605	0.407318691548476\\
1.61	0.405230976245547\\
1.615	0.403154537828323\\
1.62	0.401089306479623\\
1.625	0.399035212950817\\
1.63	0.396992188555224\\
1.635	0.394960165161618\\
1.64	0.392939075187836\\
1.645	0.390928851594474\\
1.65	0.388929427878694\\
1.655	0.386940738068113\\
1.66	0.384962716714791\\
1.665	0.382995298889307\\
1.67	0.38103842017493\\
1.675	0.379092016661871\\
1.68	0.377156024941622\\
1.685	0.375230382101389\\
1.69	0.373315025718592\\
1.695	0.371409893855462\\
1.7	0.369514925053711\\
1.705	0.367630058329277\\
1.71	0.365755233167154\\
1.715	0.363890389516294\\
1.72	0.362035467784586\\
1.725	0.360190408833901\\
1.73	0.358355153975219\\
1.735	0.356529644963822\\
1.74	0.354713823994552\\
1.745	0.352907633697147\\
1.75	0.351111017131634\\
1.755	0.349323917783794\\
1.76	0.347546279560693\\
1.765	0.345778046786266\\
1.77	0.344019164196982\\
1.775	0.342269576937549\\
1.78	0.340529230556695\\
1.785	0.338798071003003\\
1.79	0.337076044620805\\
1.795	0.335363098146129\\
1.8	0.333659178702709\\
1.805	0.331964233798046\\
1.81	0.330278211319525\\
1.815	0.328601059530586\\
1.82	0.326932727066946\\
1.825	0.325273162932872\\
1.83	0.323622316497506\\
1.835	0.321980137491245\\
1.84	0.32034657600216\\
1.845	0.318721582472469\\
1.85	0.31710510769506\\
1.855	0.315497102810057\\
1.86	0.313897519301429\\
1.865	0.312306308993654\\
1.87	0.310723424048416\\
1.875	0.309148816961356\\
1.88	0.307582440558856\\
1.885	0.306024247994874\\
1.89	0.304474192747815\\
1.895	0.302932228617448\\
1.9	0.301398309721855\\
1.905	0.299872390494431\\
1.91	0.29835442568091\\
1.915	0.296844370336445\\
1.92	0.295342179822705\\
1.925	0.293847809805033\\
1.93	0.292361216249621\\
1.935	0.29088235542073\\
1.94	0.289411183877947\\
1.945	0.287947658473467\\
1.95	0.286491736349425\\
1.955	0.285043374935243\\
1.96	0.28360253194503\\
1.965	0.282169165374993\\
1.97	0.280743233500904\\
1.975	0.279324694875577\\
1.98	0.27791350832639\\
1.985	0.27650963295283\\
1.99	0.275113028124077\\
1.995	0.273723653476608\\
2	0.272341468911832\\
2.005	0.270966434593763\\
2.01	0.26959851094671\\
2.015	0.268237658653003\\
2.02	0.266883838650737\\
2.025	0.265537012131557\\
2.03	0.264197140538455\\
2.035	0.262864185563609\\
2.04	0.261538109146227\\
2.045	0.260218873470441\\
2.05	0.258906440963204\\
2.055	0.25760077429223\\
2.06	0.256301836363941\\
2.065	0.255009590321455\\
2.07	0.253723999542585\\
2.075	0.252445027637866\\
2.08	0.251172638448607\\
2.085	0.249906796044962\\
2.09	0.248647464724022\\
2.095	0.247394609007938\\
2.1	0.246148193642054\\
2.105	0.244908183593069\\
2.11	0.243674544047213\\
2.115	0.242447240408458\\
2.12	0.241226238296729\\
2.125	0.240011503546151\\
2.13	0.238803002203311\\
2.135	0.237600700525538\\
2.14	0.236404564979201\\
2.145	0.235214562238033\\
2.15	0.234030659181464\\
2.155	0.23285282289298\\
2.16	0.231681020658499\\
2.165	0.23051521996476\\
2.17	0.229355388497734\\
2.175	0.228201494141053\\
2.18	0.227053504974454\\
2.185	0.225911389272237\\
2.19	0.22477511550175\\
2.195	0.223644652321878\\
2.2	0.222519968581555\\
2.205	0.221401033318293\\
2.21	0.220287815756721\\
2.215	0.219180285307146\\
2.22	0.218078411564126\\
2.225	0.216982164305056\\
2.23	0.215891513488775\\
2.235	0.21480642925418\\
2.24	0.213726881918863\\
2.245	0.212652841977757\\
2.25	0.211584280101791\\
2.255	0.210521167136575\\
2.26	0.20946347410108\\
2.265	0.208411172186344\\
2.27	0.207364232754188\\
2.275	0.206322627335943\\
2.28	0.205286327631193\\
2.285	0.204255305506527\\
2.29	0.20322953299431\\
2.295	0.20220898229146\\
2.3	0.201193625758239\\
2.305	0.200183435917062\\
2.31	0.199178385451306\\
2.315	0.198178447204144\\
2.32	0.197183594177382\\
2.325	0.19619379953031\\
2.33	0.195209036578564\\
2.335	0.194229278793001\\
2.34	0.193254499798585\\
2.345	0.192284673373278\\
2.35	0.191319773446952\\
2.355	0.190359774100301\\
2.36	0.189404649563776\\
2.365	0.188454374216513\\
2.37	0.187508922585289\\
2.375	0.186568269343479\\
2.38	0.185632389310022\\
2.385	0.1847012574484\\
2.39	0.18377484886563\\
2.395	0.182853138811254\\
2.4	0.181936102676352\\
2.405	0.181023715992558\\
2.41	0.180115954431082\\
2.415	0.179212793801749\\
2.42	0.178314210052039\\
2.425	0.177420179266144\\
2.43	0.176530677664025\\
2.435	0.175645681600487\\
2.44	0.174765167564252\\
2.445	0.173889112177052\\
2.45	0.173017492192721\\
2.455	0.1721502844963\\
2.46	0.171287466103149\\
2.465	0.170429014158068\\
2.47	0.169574905934424\\
2.475	0.168725118833285\\
2.48	0.167879630382568\\
2.485	0.167038418236189\\
2.49	0.166201460173219\\
2.495	0.165368734097054\\
2.5	0.164540218034588\\
2.505	0.163715890135392\\
2.51	0.162895728670907\\
2.515	0.162079712033635\\
2.52	0.161267818736342\\
2.525	0.160460027411273\\
2.53	0.159656316809362\\
2.535	0.158856665799457\\
2.54	0.158061053367554\\
2.545	0.157269458616027\\
2.55	0.156481860762877\\
2.555	0.155698239140979\\
2.56	0.15491857319734\\
2.565	0.154142842492359\\
2.57	0.153371026699097\\
2.575	0.152603105602553\\
2.58	0.151839059098943\\
2.585	0.151078867194991\\
2.59	0.150322510007217\\
2.595	0.14956996776124\\
2.6	0.148821220791083\\
2.605	0.14807624953848\\
2.61	0.147335034552196\\
2.615	0.14659755648735\\
2.62	0.14586379610474\\
2.625	0.145133734270174\\
2.63	0.144407351953818\\
2.635	0.143684630229529\\
2.64	0.142965550274212\\
2.645	0.142250093367172\\
2.65	0.141538240889476\\
2.655	0.140829974323313\\
2.66	0.140125275251372\\
2.665	0.139424125356211\\
2.67	0.138726506419642\\
2.675	0.138032400322112\\
2.68	0.137341789042097\\
2.685	0.136654654655495\\
2.69	0.135970979335025\\
2.695	0.135290745349637\\
2.7	0.134613935063914\\
2.705	0.133940530937491\\
2.71	0.133270515524471\\
2.715	0.13260387147285\\
2.72	0.131940581523943\\
2.725	0.131280628511817\\
2.73	0.130623995362728\\
2.735	0.12997066509456\\
2.74	0.129320620816271\\
2.745	0.128673845727342\\
2.75	0.128030323117233\\
2.755	0.127390036364837\\
2.76	0.126752968937945\\
2.765	0.12611910439271\\
2.77	0.125488426373116\\
2.775	0.124860918610457\\
2.78	0.124236564922811\\
2.785	0.123615349214519\\
2.79	0.12299725547568\\
2.795	0.122382267781632\\
2.8	0.121770370292448\\
2.805	0.121161547252435\\
2.81	0.120555782989633\\
2.815	0.119953061915323\\
2.82	0.119353368523528\\
2.825	0.118756687390533\\
2.83	0.118163003174396\\
2.835	0.117572300614468\\
2.84	0.116984564530917\\
2.845	0.116399779824249\\
2.85	0.115817931474846\\
2.855	0.11523900454249\\
2.86	0.114662984165906\\
2.865	0.114089855562296\\
2.87	0.113519604026887\\
2.875	0.112952214932475\\
2.88	0.112387673728972\\
2.885	0.111825965942963\\
2.89	0.111267077177259\\
2.895	0.110710993110458\\
2.9	0.110157699496503\\
2.905	0.109607182164253\\
2.91	0.109059427017047\\
2.915	0.108514420032276\\
2.92	0.107972147260958\\
2.925	0.107432594827317\\
2.93	0.106895748928359\\
2.935	0.106361595833458\\
2.94	0.105830121883942\\
2.945	0.105301313492681\\
2.95	0.104775157143681\\
2.955	0.104251639391673\\
2.96	0.103730746861717\\
2.965	0.103212466248799\\
2.97	0.102696784317432\\
2.975	0.102183687901266\\
2.98	0.101673163902692\\
2.985	0.101165199292457\\
2.99	0.100659781109272\\
2.995	0.100156896459435\\
3	0.0996565325164437\\
3.005	0.0991586765206203\\
3.01	0.0986633157787349\\
3.015	0.0981704376636306\\
3.02	0.0976800296138538\\
3.025	0.0971920791332851\\
3.03	0.0967065737907727\\
3.035	0.0962235012197692\\
3.04	0.0957428491179706\\
3.045	0.0952646052469561\\
3.05	0.0947887574318332\\
3.055	0.0943152935608821\\
3.06	0.0938442015852047\\
3.065	0.0933754695183746\\
3.07	0.0929090854360904\\
3.075	0.0924450374758294\\
3.08	0.0919833138365067\\
3.085	0.0915239027781332\\
3.09	0.0910667926214781\\
3.095	0.0906119717477324\\
3.1	0.0901594285981749\\
3.105	0.0897091516738409\\
3.11	0.0892611295351922\\
3.115	0.0888153508017896\\
3.12	0.0883718041519677\\
3.125	0.087930478322511\\
3.13	0.0874913621083338\\
3.135	0.0870544443621594\\
3.14	0.0866197139942044\\
3.145	0.0861871599718625\\
3.15	0.0857567713193919\\
3.155	0.0853285371176042\\
3.16	0.0849024465035543\\
3.165	0.0844784886702346\\
3.17	0.0840566528662682\\
3.175	0.0836369283956064\\
3.18	0.083219304617227\\
3.185	0.0828037709448348\\
3.19	0.0823903168465632\\
3.195	0.0819789318446794\\
3.2	0.0815696055152893\\
3.205	0.081162327488046\\
3.21	0.080757087445859\\
3.215	0.080353875124606\\
3.22	0.0799526803128459\\
3.225	0.0795534928515332\\
3.23	0.0791563026337358\\
3.235	0.0787610996043525\\
3.24	0.0783678737598338\\
3.245	0.0779766151479034\\
3.25	0.0775873138672821\\
3.255	0.0771999600674129\\
3.26	0.0768145439481877\\
3.265	0.0764310557596772\\
3.27	0.0760494858018592\\
3.275	0.0756698244243522\\
3.28	0.0752920620261481\\
3.285	0.0749161890553474\\
3.29	0.0745421960088966\\
3.295	0.0741700734323258\\
3.3	0.0737998119194886\\
3.305	0.0734314021123043\\
3.31	0.0730648347005006\\
3.315	0.0727001004213577\\
3.32	0.0723371900594549\\
3.325	0.0719760944464183\\
3.33	0.0716168044606697\\
3.335	0.0712593110271771\\
3.34	0.0709036051172073\\
3.345	0.0705496777480788\\
3.35	0.0701975199829176\\
3.355	0.0698471229304129\\
3.36	0.0694984777445758\\
3.365	0.0691515756244979\\
3.37	0.0688064078141129\\
3.375	0.0684629656019581\\
3.38	0.0681212403209384\\
3.385	0.067781223348091\\
3.39	0.0674429061043519\\
3.395	0.0671062800543238\\
3.4	0.0667713367060447\\
3.405	0.066438067610759\\
3.41	0.0661064643626885\\
3.415	0.0657765185988062\\
3.42	0.0654482219986102\\
3.425	0.0651215662838999\\
3.43	0.0647965432185524\\
3.435	0.0644731446083017\\
3.44	0.0641513623005171\\
3.445	0.0638311881839854\\
3.45	0.0635126141886923\\
3.455	0.0631956322856055\\
3.46	0.0628802344864603\\
3.465	0.0625664128435447\\
3.47	0.0622541594494865\\
3.475	0.0619434664370426\\
3.48	0.061634325978887\\
3.485	0.0613267302874029\\
3.49	0.0610206716144742\\
3.495	0.060716142251278\\
3.5	0.0604131345280802\\
3.505	0.0601116408140297\\
3.51	0.059811653516956\\
3.515	0.0595131650831663\\
3.52	0.0592161679972449\\
3.525	0.0589206547818532\\
3.53	0.0586266179975309\\
3.535	0.0583340502424981\\
3.54	0.0580429441524593\\
3.545	0.0577532924004073\\
3.55	0.0574650876964296\\
3.555	0.0571783227875144\\
3.56	0.0568929904573591\\
3.565	0.0566090835261791\\
3.57	0.0563265948505172\\
3.575	0.0560455173230558\\
3.58	0.0557658438724277\\
3.585	0.0554875674630306\\
3.59	0.0552106810948402\\
3.595	0.0549351778032262\\
3.6	0.0546610506587684\\
3.605	0.0543882927670737\\
3.61	0.0541168972685949\\
3.615	0.0538468573384497\\
3.62	0.0535781661862414\\
3.625	0.0533108170558796\\
3.63	0.0530448032254035\\
3.635	0.0527801180068043\\
3.64	0.0525167547458499\\
3.645	0.0522547068219102\\
3.65	0.0519939676477834\\
3.655	0.0517345306695227\\
3.66	0.0514763893662655\\
3.665	0.0512195372500609\\
3.67	0.0509639678657012\\
3.675	0.0507096747905523\\
3.68	0.0504566516343855\\
3.685	0.0502048920392101\\
3.69	0.0499543896791081\\
3.695	0.0497051382600676\\
3.7	0.0494571315198193\\
3.705	0.0492103632276723\\
3.71	0.0489648271843521\\
3.715	0.0487205172218383\\
3.72	0.0484774272032038\\
3.725	0.0482355510224551\\
3.73	0.0479948826043725\\
3.735	0.047755415904353\\
3.74	0.0475171449082515\\
3.745	0.0472800636322254\\
3.75	0.0470441661225782\\
3.755	0.0468094464556047\\
3.76	0.0465758987374375\\
3.765	0.0463435171038932\\
3.77	0.0461122957203204\\
3.775	0.0458822287814482\\
3.78	0.045653310511235\\
3.785	0.0454255351627194\\
3.79	0.0451988970178704\\
3.795	0.0449733903874394\\
3.8	0.0447490096108129\\
3.805	0.0445257490558655\\
3.81	0.044303603118814\\
3.815	0.0440825662240728\\
3.82	0.0438626328241086\\
3.825	0.0436437973992978\\
3.83	0.0434260544577834\\
3.835	0.0432093985353323\\
3.84	0.0429938241951955\\
3.845	0.042779326027966\\
3.85	0.04256589865144\\
3.855	0.0423535367104777\\
3.86	0.0421422348768651\\
3.865	0.0419319878491761\\
3.87	0.0417227903526362\\
3.875	0.0415146371389861\\
3.88	0.0413075229863465\\
3.885	0.0411014426990834\\
3.89	0.0408963911076744\\
3.895	0.0406923630685753\\
3.9	0.0404893534640877\\
3.905	0.0402873572022272\\
3.91	0.0400863692165925\\
3.915	0.0398863844662345\\
3.92	0.0396873979355276\\
3.925	0.0394894046340394\\
3.93	0.0392923995964037\\
3.935	0.0390963778821917\\
3.94	0.0389013345757861\\
3.945	0.038707264786254\\
3.95	0.0385141636472215\\
3.955	0.038322026316749\\
3.96	0.0381308479772067\\
3.965	0.037940623835151\\
3.97	0.0377513491212014\\
3.975	0.0375630190899183\\
3.98	0.0373756290196814\\
3.985	0.0371891742125682\\
3.99	0.0370036499942343\\
3.995	0.0368190517137929\\
4	0.0366353747436963\\
4.005	0.0364526144796167\\
4.01	0.0362707663403287\\
4.015	0.0360898257675922\\
4.02	0.0359097882260352\\
4.025	0.0357306492030383\\
4.03	0.0355524042086187\\
4.035	0.035375048775316\\
4.04	0.0351985784580772\\
4.045	0.0350229888341438\\
4.05	0.0348482755029385\\
4.055	0.034674434085952\\
4.06	0.0345014602266326\\
4.065	0.0343293495902738\\
4.07	0.0341580978639038\\
4.075	0.0339877007561757\\
4.08	0.0338181539972574\\
4.085	0.0336494533387235\\
4.09	0.033481594553446\\
4.095	0.0333145734354867\\
4.1	0.0331483857999905\\
4.105	0.0329830274830778\\
4.11	0.032818494341739\\
4.115	0.0326547822537286\\
4.12	0.03249188711746\\
4.125	0.0323298048519013\\
4.13	0.0321685313964711\\
4.135	0.0320080627109354\\
4.14	0.0318483947753043\\
4.145	0.0316895235897297\\
4.15	0.0315314451744034\\
4.155	0.0313741555694566\\
4.16	0.0312176508348579\\
4.165	0.0310619270503137\\
4.17	0.0309069803151686\\
4.175	0.0307528067483058\\
4.18	0.0305994024880488\\
4.185	0.0304467636920628\\
4.19	0.030294886537257\\
4.195	0.0301437672196881\\
4.2	0.0299934019544627\\
4.205	0.0298437869756417\\
4.21	0.0296949185361446\\
4.215	0.0295467929076541\\
4.22	0.0293994063805214\\
4.225	0.0292527552636723\\
4.23	0.029106835884513\\
4.235	0.0289616445888374\\
4.24	0.0288171777407339\\
4.245	0.0286734317224934\\
4.25	0.0285304029345175\\
4.255	0.0283880877952267\\
4.26	0.0282464827409702\\
4.265	0.0281055842259354\\
4.27	0.0279653887220574\\
4.275	0.0278258927189302\\
4.28	0.0276870927237173\\
4.285	0.0275489852610634\\
4.29	0.0274115668730062\\
4.295	0.0272748341188888\\
4.3	0.0271387835752722\\
4.305	0.0270034118358493\\
4.31	0.0268687155113576\\
4.315	0.0267346912294947\\
4.32	0.0266013356348311\\
4.325	0.0264686453887272\\
4.33	0.026336617169247\\
4.335	0.0262052476710753\\
4.34	0.0260745336054332\\
4.345	0.0259444716999955\\
4.35	0.0258150586988074\\
4.355	0.0256862913622021\\
4.36	0.0255581664667192\\
4.365	0.025430680805023\\
4.37	0.0253038311858213\\
4.375	0.0251776144337845\\
4.38	0.0250520273894659\\
4.385	0.0249270669092211\\
4.39	0.0248027298651289\\
4.395	0.0246790131449122\\
4.4	0.0245559136518593\\
4.405	0.0244334283047457\\
4.41	0.0243115540377557\\
4.415	0.024190287800406\\
4.42	0.0240696265574677\\
4.425	0.0239495672888898\\
4.43	0.0238301069897231\\
4.435	0.0237112426700442\\
4.44	0.0235929713548801\\
4.445	0.0234752900841328\\
4.45	0.023358195912505\\
4.455	0.0232416859094249\\
4.46	0.0231257571589735\\
4.465	0.0230104067598098\\
4.47	0.0228956318250979\\
4.475	0.0227814294824345\\
4.48	0.0226677968737763\\
4.485	0.0225547311553674\\
4.49	0.0224422294976679\\
4.495	0.022330289085283\\
4.5	0.0222189071168908\\
4.505	0.0221080808051727\\
4.51	0.0219978073767427\\
4.515	0.0218880840720773\\
4.52	0.0217789081454462\\
4.525	0.0216702768648427\\
4.53	0.0215621875119154\\
4.535	0.021454637381899\\
4.54	0.0213476237835465\\
4.545	0.0212411440390613\\
4.55	0.0211351954840296\\
4.555	0.0210297754673534\\
4.56	0.0209248813511837\\
4.565	0.0208205105108537\\
4.57	0.020716660334813\\
4.575	0.0206133282245617\\
4.58	0.0205105115945852\\
4.585	0.0204082078722881\\
4.59	0.0203064144979308\\
4.595	0.020205128924564\\
4.6	0.020104348617965\\
4.605	0.0200040710565734\\
4.61	0.0199042937314284\\
4.615	0.0198050141461049\\
4.62	0.0197062298166508\\
4.625	0.0196079382715249\\
4.63	0.0195101370515342\\
4.635	0.019412823709772\\
4.64	0.0193159958115565\\
4.645	0.0192196509343697\\
4.65	0.0191237866677958\\
4.655	0.0190284006134611\\
4.66	0.0189334903849737\\
4.665	0.0188390536078628\\
4.67	0.0187450879195197\\
4.675	0.0186515909691378\\
4.68	0.0185585604176539\\
4.685	0.018465993937689\\
4.69	0.0183738892134898\\
4.695	0.0182822439408708\\
4.7	0.0181910558271558\\
4.705	0.0181003225911208\\
4.71	0.0180100419629362\\
4.715	0.0179202116841097\\
4.72	0.01783082950743\\
4.725	0.0177418931969097\\
4.73	0.0176534005277296\\
4.735	0.0175653492861823\\
4.74	0.0174777372696167\\
4.745	0.0173905622863829\\
4.75	0.0173038221557769\\
4.755	0.0172175147079853\\
4.76	0.0171316377840319\\
4.765	0.0170461892357223\\
4.77	0.0169611669255904\\
4.775	0.016876568726845\\
4.78	0.0167923925233157\\
4.785	0.0167086362093999\\
4.79	0.0166252976900106\\
4.795	0.0165423748805225\\
4.8	0.0164598657067208\\
4.805	0.0163777681047485\\
4.81	0.0162960800210542\\
4.815	0.0162147994123418\\
4.82	0.0161339242455175\\
4.825	0.01605345249764\\
4.83	0.0159733821558694\\
4.835	0.0158937112174159\\
4.84	0.0158144376894906\\
4.845	0.0157355595892547\\
4.85	0.0156570749437702\\
4.855	0.0155789817899499\\
4.86	0.0155012781745084\\
4.865	0.0154239621539126\\
4.87	0.0153470317943341\\
4.875	0.0152704851715988\\
4.88	0.0151943203711403\\
4.885	0.0151185354879511\\
4.89	0.0150431286265343\\
4.895	0.014968097900857\\
4.9	0.0148934414343024\\
4.905	0.0148191573596225\\
4.91	0.0147452438188921\\
4.915	0.0146716989634609\\
4.92	0.0145985209539082\\
4.925	0.0145257079599961\\
4.93	0.014453258160624\\
4.935	0.0143811697437823\\
4.94	0.0143094409065079\\
4.945	0.0142380698548376\\
4.95	0.0141670548037647\\
4.955	0.0140963939771928\\
4.96	0.0140260856078917\\
4.965	0.0139561279374536\\
4.97	0.0138865192162483\\
4.975	0.0138172577033794\\
4.98	0.0137483416666412\\
4.985	0.0136797693824743\\
4.99	0.0136115391359234\\
4.995	0.0135436492205932\\
5	0.0134760979386066\\
5.005	0.0134088836005616\\
5.01	0.0133420045254887\\
5.015	0.0132754590408094\\
5.02	0.0132092454822937\\
5.025	0.0131433621940187\\
5.03	0.0130778075283266\\
5.035	0.0130125798457839\\
5.04	0.0129476775151403\\
5.045	0.0128830989132871\\
5.05	0.0128188424252175\\
5.055	0.012754906443985\\
5.06	0.0126912893706643\\
5.065	0.0126279896143103\\
5.07	0.0125650055919182\\
5.075	0.0125023357283847\\
5.08	0.0124399784564677\\
5.085	0.0123779322167471\\
5.09	0.0123161954575859\\
5.095	0.0122547666350913\\
5.1	0.012193644213076\\
5.105	0.0121328266630194\\
5.11	0.0120723124640299\\
5.115	0.0120121001028059\\
5.12	0.0119521880735987\\
5.125	0.0118925748781746\\
5.13	0.0118332590257765\\
5.135	0.0117742390330878\\
5.14	0.0117155134241945\\
5.145	0.0116570807305481\\
5.15	0.0115989394909292\\
5.155	0.0115410882514107\\
5.16	0.0114835255653215\\
5.165	0.0114262499932098\\
5.17	0.0113692601028076\\
5.175	0.0113125544689945\\
5.18	0.0112561316737619\\
5.185	0.0111999903061778\\
5.19	0.0111441289623512\\
5.195	0.0110885462453968\\
5.2	0.0110332407654006\\
5.205	0.0109782111393842\\
5.21	0.0109234559912708\\
5.215	0.0108689739518505\\
5.22	0.0108147636587462\\
5.225	0.0107608237563789\\
5.23	0.0107071528959344\\
5.235	0.0106537497353293\\
5.24	0.0106006129391771\\
5.245	0.0105477411787551\\
5.25	0.0104951331319708\\
5.255	0.0104427874833292\\
5.26	0.0103907029238995\\
5.265	0.0103388781512824\\
5.27	0.0102873118695775\\
5.275	0.0102360027893507\\
5.28	0.0101849496276022\\
5.285	0.0101341511077343\\
5.29	0.0100836059595191\\
5.295	0.010033312919067\\
5.3	0.00998327072879492\\
5.305	0.00993347813739485\\
5.31	0.00988393389980241\\
5.315	0.00983463677716612\\
5.32	0.00978558553681543\\
5.325	0.00973677895223093\\
5.33	0.00968821580301282\\
5.335	0.00963989487485094\\
5.34	0.00959181495949363\\
5.345	0.00954397485471851\\
5.35	0.00949637336430126\\
5.355	0.00944900929798663\\
5.36	0.00940188147145772\\
5.365	0.00935498870630725\\
5.37	0.00930832983000728\\
5.375	0.00926190367588046\\
5.38	0.00921570908307036\\
5.385	0.0091697448965125\\
5.39	0.00912400996690554\\
5.395	0.00907850315068271\\
5.4	0.00903322330998276\\
5.405	0.00898816931262147\\
5.41	0.00894334003206366\\
5.415	0.00889873434739493\\
5.42	0.00885435114329325\\
5.425	0.0088101893100014\\
5.43	0.00876624774329891\\
5.435	0.00872252534447459\\
5.44	0.00867902102029916\\
5.445	0.00863573368299753\\
5.45	0.00859266225022169\\
5.455	0.00854980564502394\\
5.46	0.00850716279582937\\
5.465	0.00846473263640943\\
5.47	0.00842251410585515\\
5.475	0.00838050614855051\\
5.48	0.00833870771414631\\
5.485	0.00829711775753329\\
5.49	0.00825573523881662\\
5.495	0.00821455912328918\\
5.5	0.00817358838140669\\
};
\end{axis}
\end{tikzpicture}%
		\caption{$\tilde{\Phi}(x)$}
		\label{fig:info_uni}
	\end{figure}
	\end{columns}
	\vfill
	Min Sum update was implemented too, by changing the function in the check node
	$$
		LLR_{+\rightarrow j} = \min_{i \ne j} \left\{ |LLR_{i \rightarrow +}| \right\} \Pi_{i \ne j} \mbox{sign} \left( LLR_{i \rightarrow +} \right)
	$$

\end{frame}

\begin{frame}{Marginalization}
	The marginalization is carried out between leaf nodes $g_l$ and variable nodes $=_l$, thus
	$$
		\hat{x} =
		\begin{cases}
			0, &\text{ if }LLR_{g_l \rightarrow x} + LLR_{=_l \rightarrow x} \ge 0\\
			1, &\text{ otherwise }
		\end{cases}
	$$
\end{frame}

\begin{frame}{Initialization and Schedule}
	LDPC codes have cycles. Therefore to decode we need
	\begin{itemize}
		\item \textbf{Initialization}: Variable nodes are initialized with the leaf node LLR
		\item \textbf{Schedule}: 
		\begin{enumerate}
			\item Run message passing on check nodes $+$ and update their outgoing LLRs
			\item Run message passing on variable nodes $=$
			\item Marginalize: \textbf{if} a codeword is found or if the maximum number of attempts is reached \textit{stop}, \textbf{else} go to 1
		\end{enumerate}
	\end{itemize}
\end{frame}

\begin{frame}{C++ implementation 1/3}
	The message passing decoder was implemented using the \textbf{flexibility} offered by \textit{Object Oriented Programming} (OOP). 
	\begin{itemize}
		\item \textbf{\texttt{VariableNode} class} represents a single variable node, and it is initialized with a position in the standard matrix $\mathbf{S}$ and the index of its 7 check nodes,
		\item \textbf{\texttt{CheckNode} class} represents a single check node, it knows to which variable node is connected to,
		\item \textbf{\texttt{LdpcDecoder} class} contains a vector of variable nodes, a vector of check nodes and a vector of received LLR. It handles initialization, the update schedule and marginalization.
	\end{itemize}
\end{frame}

\begin{frame}{C++ implementation 2/3} 
	\begin{itemize}
		\item \texttt{LdpcDecoder} is initialized once per simulation campaign,
		\item The $\tilde{\Phi}(x)$ function is clipped to \texttt{infinity()} for $x < 10^{-300}$ and to 0 for $x > 38$,
		\item The Sum Product update computes once all the $\tilde{\Phi}$ values, sums them and the subtracts the outgoing $\tilde{\Phi}$,
		\item \textbf{Testing}: update in variable and check node is tested to check if the results obtained are as expected.
	\end{itemize}
\end{frame}

\begin{frame}{C++ implementation 3/3} 
	\begin{itemize}
		\item For each simulation, a single noise vector is generated, and scaled by $\sigma_w$ for each $\frac{E_b}{N_0}$ with
		$$
			\sigma_w = \sqrt{\frac{1}{\sqrt{2 \frac{E_b}{N_0} R}}}
		$$
		\item The decoding for each $\frac{E_b}{N_0}$ is launched in a separate thread, to parallelize computations
	\end{itemize}
\end{frame}

\begin{frame}{Results: BER for different max number of iterations}
	\begin{figure}[t]
		\centering
		\setlength\fheight{0.5\textwidth}
		\setlength\fwidth{0.8\textwidth}
		% This file was created by matlab2tikz.
%
%The latest updates can be retrieved from
%  http://www.mathworks.com/matlabcentral/fileexchange/22022-matlab2tikz-matlab2tikz
%where you can also make suggestions and rate matlab2tikz.
%
\definecolor{mycolor1}{rgb}{0.00000,0.44700,0.74100}%
\definecolor{mycolor2}{rgb}{0.85000,0.32500,0.09800}%
\definecolor{mycolor3}{rgb}{0.92900,0.69400,0.12500}%
\definecolor{mycolor4}{rgb}{0.49400,0.18400,0.55600}%
\definecolor{mycolor5}{rgb}{0.46600,0.67400,0.18800}%
\definecolor{mycolor6}{rgb}{0.30100,0.74500,0.93300}%
%
\begin{tikzpicture}
\tikzstyle{every node}=[font=\small]
\begin{axis}[%
grid style={line width=.1pt, draw=gray!10},
width=0.951\fwidth,
height=\fheight,
at={(0\fwidth,0\fheight)},
scale only axis,
xmin=4.5,
xmax=4.95,
xlabel={\scriptsize{$\dfrac{E_b}{N_0}$ [dB]}},
xmajorgrids,
ymode=log,
ymin=5e-08,
ymax=0.02,
yminorticks=true,
ylabel={BER},
ymajorgrids,
yminorgrids,
axis background/.style={fill=white},
legend style={at={(0.01,0.01)},legend cell align=left,align=left,anchor=south west,draw=white!15!black}
]
\addplot [color=mycolor1,solid,mark=x,mark options={solid}]
  table[row sep=crcr]{%
4.5	0.00864212866108787\\
4.6	0.00454743070083682\\
4.65	0.00243066814853556\\
4.7	0.000963143959205021\\
4.72	0.000585470057531381\\
4.74	0.000337800732217573\\
4.76	0.000187810538702929\\
4.78	8.90134675732218e-05\\
4.8	4.51654027196653e-05\\
4.82	2.26399058577406e-05\\
4.84	1.10445541317992e-05\\
4.86	4.85257583682008e-06\\
4.88	1.90490978033473e-06\\
4.9	6.668410041841e-07\\
4.92	2.59256128619216e-07\\
4.94	1.00661770051425e-07\\
};
\addlegendentry{100};

\addplot [color=mycolor2,solid,mark=triangle,mark options={solid}]
  table[row sep=crcr]{%
3	0.026578190376569\\
3.5	0.0202333943514644\\
4	0.0146142782426778\\
4.2	0.0124888859832636\\
4.6	0.00504118723849372\\
4.7	0.000967148274058577\\
4.72	0.000574725418410042\\
4.74	0.000337637290794979\\
4.76	0.000211252496196272\\
4.78	0.000111277802246201\\
4.8	5.48733070909491e-05\\
4.82	2.55995062963637e-05\\
4.84	8.61738202208656e-06\\
4.86	3.94745416446629e-06\\
4.88	1.53821727435744e-06\\
4.9	5.7727510460251e-07\\
4.92	2.33232568697964e-07\\
4.94	6.23734050889306e-08\\
};
\addlegendentry{50};

\addplot [color=mycolor3,solid,mark=diamond,mark options={solid}]
  table[row sep=crcr]{%
3	0.0265589042887029\\
4	0.0146401019874477\\
4.2	0.0125129118723849\\
4.5	0.00871959989539749\\
4.6	0.00566471626569038\\
4.7	0.00136375820178775\\
4.8	0.000103705068467098\\
4.9	2.35483837886619e-06\\
4.94	5.54883629707113e-07\\
5	0\\
};
\addlegendentry{20};

\addplot [color=mycolor4,solid,mark=asterisk,mark options={solid}]
  table[row sep=crcr]{%
2	0.0423006014644351\\
3	0.0264922201882845\\
4	0.0145681877615063\\
4.5	0.00888892520920502\\
4.6	0.00648274058577406\\
4.7	0.00263650627615063\\
4.8	0.000399875784518828\\
4.9	2.23855315709395e-05\\
5	4.19004374286801e-07\\
5.1	0\\
6	0\\
};
\addlegendentry{10};

\addplot [color=mycolor5,solid,mark=o,mark options={solid}]
  table[row sep=crcr]{%
2	0.042520920502092\\
3	0.0266112055439331\\
4	0.0146956720711297\\
4.5	0.00969730648535565\\
5	0.00528896443514644\\
5.5	0.00200509937238494\\
6	0.000474960774058577\\
6.5	5.91004184100418e-05\\
7	4.21678870292887e-06\\
};
\addlegendentry{1};

\addplot [color=mycolor6,solid,mark=triangle,mark options={solid,rotate=180}]
  table[row sep=crcr]{%
0	0.0786496035251426\\
1	0.0562819519765415\\
2	0.037506128358926\\
3	0.0228784075610853\\
4	0.0125008180407376\\
5	0.00595386714777866\\
6	0.00238829078093281\\
7	0.000772674815378444\\
8	0.000190907774075993\\
9	3.36272284196176e-05\\
10	3.87210821552205e-06\\
};
\addlegendentry{Uncoded};

\end{axis}
\end{tikzpicture}%
		\caption{\small{BER for different number of iterations}}
	\end{figure}
\end{frame}

\begin{frame}{The waterfall behavior}
	\begin{figure}[t]
		\centering
		\setlength\fheight{0.5\textwidth}
		\setlength\fwidth{0.8\textwidth}
		% This file was created by matlab2tikz.
%
%The latest updates can be retrieved from
%  http://www.mathworks.com/matlabcentral/fileexchange/22022-matlab2tikz-matlab2tikz
%where you can also make suggestions and rate matlab2tikz.
%
\definecolor{mycolor1}{rgb}{0.00000,0.44700,0.74100}%
\definecolor{mycolor2}{rgb}{0.85000,0.32500,0.09800}%
\definecolor{mycolor3}{rgb}{0.92900,0.69400,0.12500}%
\definecolor{mycolor4}{rgb}{0.49400,0.18400,0.55600}%
\definecolor{mycolor5}{rgb}{0.46600,0.67400,0.18800}%
\definecolor{mycolor6}{rgb}{0.30100,0.74500,0.93300}%
%
\begin{tikzpicture}
\tikzstyle{every node}=[font=\small]
\begin{axis}[%
grid style={line width=.1pt, draw=gray!10},
width=0.951\fwidth,
height=\fheight,
at={(0\fwidth,0\fheight)},
scale only axis,
xmin=3.5,
xmax=6,
xlabel={\scriptsize{$\dfrac{E_b}{N_0}$ [dB]}},
xmajorgrids,
ymode=log,
ymin=5e-08,
ymax=0.02,
yminorticks=true,
ylabel={BER},
ymajorgrids,
yminorgrids,
axis background/.style={fill=white},
legend style={at={(0.01,0.01)},legend cell align=left,align=left,anchor=south west,draw=white!15!black}
]
\addplot [color=mycolor1,solid,mark=x,mark options={solid}]
  table[row sep=crcr]{%
4.5	0.00864212866108787\\
4.6	0.00454743070083682\\
4.65	0.00243066814853556\\
4.7	0.000963143959205021\\
4.72	0.000585470057531381\\
4.74	0.000337800732217573\\
4.76	0.000187810538702929\\
4.78	8.90134675732218e-05\\
4.8	4.51654027196653e-05\\
4.82	2.26399058577406e-05\\
4.84	1.10445541317992e-05\\
4.86	4.85257583682008e-06\\
4.88	1.90490978033473e-06\\
4.9	6.668410041841e-07\\
4.92	2.59256128619216e-07\\
4.94	1.00661770051425e-07\\
};
\addlegendentry{100};

\addplot [color=mycolor2,solid,mark=triangle,mark options={solid}]
  table[row sep=crcr]{%
3	0.026578190376569\\
3.5	0.0202333943514644\\
4	0.0146142782426778\\
4.2	0.0124888859832636\\
4.6	0.00504118723849372\\
4.7	0.000967148274058577\\
4.72	0.000574725418410042\\
4.74	0.000337637290794979\\
4.76	0.000211252496196272\\
4.78	0.000111277802246201\\
4.8	5.48733070909491e-05\\
4.82	2.55995062963637e-05\\
4.84	8.61738202208656e-06\\
4.86	3.94745416446629e-06\\
4.88	1.53821727435744e-06\\
4.9	5.7727510460251e-07\\
4.92	2.33232568697964e-07\\
4.94	6.23734050889306e-08\\
};
\addlegendentry{50};

\addplot [color=mycolor3,solid,mark=diamond,mark options={solid}]
  table[row sep=crcr]{%
3	0.0265589042887029\\
4	0.0146401019874477\\
4.2	0.0125129118723849\\
4.5	0.00871959989539749\\
4.6	0.00566471626569038\\
4.7	0.00136375820178775\\
4.8	0.000103705068467098\\
4.9	2.35483837886619e-06\\
4.94	5.54883629707113e-07\\
5	0\\
};
\addlegendentry{20};

\addplot [color=mycolor4,solid,mark=asterisk,mark options={solid}]
  table[row sep=crcr]{%
2	0.0423006014644351\\
3	0.0264922201882845\\
4	0.0145681877615063\\
4.5	0.00888892520920502\\
4.6	0.00648274058577406\\
4.7	0.00263650627615063\\
4.8	0.000399875784518828\\
4.9	2.23855315709395e-05\\
5	4.19004374286801e-07\\
5.1	0\\
6	0\\
};
\addlegendentry{10};

\addplot [color=mycolor5,solid,mark=o,mark options={solid}]
  table[row sep=crcr]{%
2	0.042520920502092\\
3	0.0266112055439331\\
4	0.0146956720711297\\
4.5	0.00969730648535565\\
5	0.00528896443514644\\
5.5	0.00200509937238494\\
6	0.000474960774058577\\
6.5	5.91004184100418e-05\\
7	4.21678870292887e-06\\
};
\addlegendentry{1};

\addplot [color=mycolor6,solid,mark=triangle,mark options={solid,rotate=180}]
  table[row sep=crcr]{%
0	0.0786496035251426\\
1	0.0562819519765415\\
2	0.037506128358926\\
3	0.0228784075610853\\
4	0.0125008180407376\\
5	0.00595386714777866\\
6	0.00238829078093281\\
7	0.000772674815378444\\
8	0.000190907774075993\\
9	3.36272284196176e-05\\
10	3.87210821552205e-06\\
};
\addlegendentry{Uncoded};

\end{axis}
\end{tikzpicture}%
		\caption{\small{BER for 50 iterations}}
	\end{figure}
\end{frame}

\begin{frame}{Packet Error Rate for 50 iterations}
	\begin{figure}[t]
		\centering
		\setlength\fheight{0.5\textwidth}
		\setlength\fwidth{0.8\textwidth}
		% This file was created by matlab2tikz.
%
%The latest updates can be retrieved from
%  http://www.mathworks.com/matlabcentral/fileexchange/22022-matlab2tikz-matlab2tikz
%where you can also make suggestions and rate matlab2tikz.
%
\definecolor{mycolor1}{rgb}{0.00000,0.44700,0.74100}%
%
\begin{tikzpicture}

\begin{axis}[%
grid style={line width=.1pt, draw=gray!10},
width=0.951\fwidth,
height=\fheight,
at={(0\fwidth,0\fheight)},
scale only axis,
xmin=3,
xmax=6,
xlabel={$\dfrac{E_b}{N_O}$ [dB]},
xmajorgrids,
ymode=log,
ymin=5e-06,
ymax=1.1,
yminorticks=true,
ylabel={PER},
ymajorgrids,
yminorgrids,
axis background/.style={fill=white},
legend style={at={(0.01,0.01)},legend cell align=left,align=left,anchor=south west,draw=white!15!black}
]
\addplot [color=mycolor1,solid,mark=triangle,mark options={solid}]
  table[row sep=crcr]{%
3	1\\
3.5	1\\
4	1\\
4.2	1\\
4.6	0.64\\
4.7	0.132\\
4.72	0.082\\
4.74	0.047\\
4.76	0.0298181818181818\\
4.78	0.0159649122807018\\
4.8	0.00780701754385965\\
4.82	0.00391304347826087\\
4.84	0.00127868852459016\\
4.86	0.000566343042071197\\
4.88	0.000228571428571429\\
4.9	9e-05\\
4.92	3.43481654957065e-05\\
4.94	1.00959111559818e-05\\
};
\addlegendentry{50 iterations};

\end{axis}
\end{tikzpicture}%
		\caption{\small{PER for 50 iterations}}
	\end{figure}
\end{frame}

\begin{frame}{Comparison with a different choice of slopes}
	\begin{figure}[t]
		\centering
		\setlength\fheight{0.5\textwidth}
		\setlength\fwidth{0.8\textwidth}
		% This file was created by matlab2tikz.
%
%The latest updates can be retrieved from
%  http://www.mathworks.com/matlabcentral/fileexchange/22022-matlab2tikz-matlab2tikz
%where you can also make suggestions and rate matlab2tikz.
%
\definecolor{mycolor1}{rgb}{0.00000,0.44700,0.74100}%
\definecolor{mycolor2}{rgb}{0.85000,0.32500,0.09800}%
\definecolor{mycolor3}{rgb}{0.92900,0.69400,0.12500}%
%
\begin{tikzpicture}
\tikzstyle{every node}=[font=\small]
\begin{axis}[%
grid style={line width=.1pt, draw=gray!10},
width=0.951\fwidth,
height=\fheight,
at={(0\fwidth,0\fheight)},
scale only axis,
xmin=4,
xmax=5.5,
xlabel={\scriptsize{$\dfrac{E_b}{N_0}$ [dB]}},
xmajorgrids,
ymode=log,
ymin=5e-08,
ymax=0.02,
yminorticks=true,
ylabel={BER},
ymajorgrids,
yminorgrids,
axis background/.style={fill=white},
legend style={at={(0.01,0.01)},legend cell align=left,align=left,anchor=south west,draw=white!15!black}
]
\addplot [color=mycolor1,solid,mark=triangle,mark options={solid}]
  table[row sep=crcr]{%
3	0.026578190376569\\
3.5	0.0202333943514644\\
4	0.0146142782426778\\
4.2	0.0124888859832636\\
4.6	0.00504118723849372\\
4.7	0.000967148274058577\\
4.72	0.000574725418410042\\
4.74	0.000337637290794979\\
4.76	0.000211252496196272\\
4.78	0.000111277802246201\\
4.8	5.48733070909491e-05\\
4.82	2.55995062963637e-05\\
4.84	8.61738202208656e-06\\
4.86	3.94745416446629e-06\\
4.88	1.53821727435744e-06\\
4.9	5.7727510460251e-07\\
4.92	2.33232568697964e-07\\
4.94	6.23734050889306e-08\\
};
\addlegendentry{1,2,3,4,5,6,7};

\addplot [color=mycolor2,solid,mark=x,mark options={solid}]
  table[row sep=crcr]{%
2	0.0422175732217573\\
3	0.0264673770920502\\
4	0.0145842050209205\\
4.5	0.00860290271966527\\
4.6	0.00457178347280335\\
4.7	0.00081194429916318\\
4.74	0.000360584466527197\\
4.78	9.47960251046025e-05\\
4.8	4.38121077405858e-05\\
4.82	2.27314330543933e-05\\
4.84	1.16272228033473e-05\\
4.86	2.34048117154812e-06\\
4.88	9.55315115062762e-07\\
4.9	7.38755230125523e-07\\
4.92	4.51098326359833e-07\\
4.94	1.3010054370816e-07\\
};
\addlegendentry{1,2,3,5,7,11,13};

\addplot [color=mycolor3,solid,mark=triangle,mark options={solid,rotate=180}]
  table[row sep=crcr]{%
0	0.0786496035251426\\
1	0.0562819519765415\\
2	0.037506128358926\\
3	0.0228784075610853\\
4	0.0125008180407376\\
5	0.00595386714777866\\
6	0.00238829078093281\\
7	0.000772674815378444\\
8	0.000190907774075993\\
9	3.36272284196176e-05\\
10	3.87210821552205e-06\\
};
\addlegendentry{Uncoded};

\end{axis}
\end{tikzpicture}%
		\caption{\small{Comparison with 2 different set of slopes}}
	\end{figure}
\end{frame}

\begin{frame}{Comparison with Min Sum}
	\begin{figure}[t]
		\centering
		\setlength\fheight{0.5\textwidth}
		\setlength\fwidth{0.8\textwidth}
		% This file was created by matlab2tikz.
%
%The latest updates can be retrieved from
%  http://www.mathworks.com/matlabcentral/fileexchange/22022-matlab2tikz-matlab2tikz
%where you can also make suggestions and rate matlab2tikz.
%
\definecolor{mycolor1}{rgb}{0.00000,0.44700,0.74100}%
\definecolor{mycolor2}{rgb}{0.85000,0.32500,0.09800}%
\definecolor{mycolor3}{rgb}{0.92900,0.69400,0.12500}%
%
\begin{tikzpicture}
\tikzstyle{every node}=[font=\small]
\begin{axis}[%
grid style={line width=.1pt, draw=gray!10},
width=0.951\fwidth,
height=\fheight,
at={(0\fwidth,0\fheight)},
scale only axis,
xmin=4,
xmax=6,
xlabel={\scriptsize{$\dfrac{E_b}{N_0}$ [dB]}},
xmajorgrids,
ymode=log,
ymin=5e-08,
ymax=0.02,
yminorticks=true,
ylabel={BER},
ymajorgrids,
yminorgrids,
axis background/.style={fill=white},
legend style={at={(0.01,0.01)},legend cell align=left,align=left,anchor=south west,draw=white!15!black}
]
\addplot [color=mycolor1,solid,mark=triangle,mark options={solid}]
  table[row sep=crcr]{%
3	0.026578190376569\\
3.5	0.0202333943514644\\
4	0.0146142782426778\\
4.2	0.0124888859832636\\
4.6	0.00504118723849372\\
4.7	0.000967148274058577\\
4.72	0.000574725418410042\\
4.74	0.000337637290794979\\
4.76	0.000211252496196272\\
4.78	0.000111277802246201\\
4.8	5.48733070909491e-05\\
4.82	2.55995062963637e-05\\
4.84	8.61738202208656e-06\\
4.86	3.94745416446629e-06\\
4.88	1.53821727435744e-06\\
4.9	5.7727510460251e-07\\
4.92	2.33232568697964e-07\\
4.94	6.23734050889306e-08\\
};
\addlegendentry{Sum Product};

\addplot [color=mycolor2,solid,mark=x,mark options={solid}]
  table[row sep=crcr]{%
2	0.044753530334728\\
2.5	0.0368893828451883\\
3	0.0301209466527197\\
3.5	0.0244073614016736\\
4	0.0194858132845188\\
4.5	0.015100679916318\\
4.7	0.0132975941422594\\
4.8	0.0124813676778243\\
4.9	0.0115268697698745\\
5	0.0102771966527197\\
5.1	0.00717624542364017\\
5.2	0.00264867939330544\\
5.3	0.00036784780334728\\
5.4	1.86519351464435e-05\\
5.5	1.24760285913529e-07\\
};
\addlegendentry{Min Sum};

\addplot [color=mycolor3,solid,mark=triangle,mark options={solid,rotate=180}]
  table[row sep=crcr]{%
0	0.0786496035251426\\
1	0.0562819519765415\\
2	0.037506128358926\\
3	0.0228784075610853\\
4	0.0125008180407376\\
5	0.00595386714777866\\
6	0.00238829078093281\\
7	0.000772674815378444\\
8	0.000190907774075993\\
9	3.36272284196176e-05\\
10	3.87210821552205e-06\\
};
\addlegendentry{Uncoded};

\end{axis}
\end{tikzpicture}%
		\caption{\small{Comparison with Min Sum}}
	\end{figure}
\end{frame}

\begin{frame}{Conclusions}
    \begin{itemize}
    	\item LDPC code for DWDM submarine systems was presented
 		\item Encoding and message passing decoding were described
 		\item The C++ implementation was detailed
 		\item Results show that with 50 iterations of Sum Product algorithm the code is close to the Shannon bound
	\end{itemize}
\end{frame}

\end{document}